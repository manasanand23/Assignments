\let\negmedspace\undefined
\let\negthickspace\undefined
\documentclass{article}
\usepackage{cite}
\usepackage{amsmath,amssymb,amsfonts,amsthm}
\usepackage{algorithmic}
\usepackage{graphicx}
\usepackage{textcomp}
\usepackage{xcolor}
\usepackage{txfonts}
\usepackage{listings}
\usepackage{enumitem}
\usepackage{tfrupee}
\usepackage{mathtools}                                                                  \usepackage{gensymb}                                                                    \usepackage{tfrupee}
\usepackage[breaklinks=true]{hyperref}                                                  \usepackage{tkz-euclide} % loads  TikZ and tkz-base
\usepackage{listings}
\usepackage{gvv}
%\usepackage{setspace}
%\usepackage{gensymb}
%\doublespacing
%\singlespacing

%\usepackage{graphicx}                                                                               %\usepackage{amssymb}
%\usepackage{relsize}
%\usepackage[cmex10]{amsmath}
%\usepackage{amsthm}
%\interdisplaylinepenalty=2500
%\savesymbol{iint}
%\usepackage{txfonts}
%\restoresymbol{TXF}{iint}
%\usepackage{wasysym}
%\usepackage{amsthm}
%\usepackage{iithtlc}
%\usepackage{mathrsfs}
%\usepackage{txfonts}
%\usepackage{stfloats}
%\usepackage{bm}
%\usepackage{cite}
%\usepackage{cases}
%\usepackage{subfig}
%\usepackage{xtab}
%\usepackage{longtable}
%\usepackage{multirow}
%\usepackage{algorithm}
%\usepackage{algpseudocode}
%\usepackage{enumitem}                                                                               %\usepackage{mathtools}                 >
%\usepackage{circuitikz}
%\usepackage{verbatim}
%\usepackage{tfrupee}
%\usepackage{stmaryrd}
%\usetkzobj{all}
%    \usepackage{color}                                            %%
%    \usepackage{array}                                            %%                                                                         %    \usepackage{longtable}                                        %%
%    \usepackage{calc}                                             %%
%    \usepackage{multirow}                                         %%
%    \usepackage{hhline}                                           %%
%    \usepackage{ifthen}                                           %%                                                                           %optionally (for landscape tables embedded in another document): %%
%    \usepackage{lscape}
%\usepackage{multicol}
%\usepackage{chngcntr}
%\usepackage{enumerate}

%\usepackage{wasysym}                                                                                %\documentclass[conference]{IEEEtran}   >
% The preceding line is only needed to identify funding in the first footnote. If that is unneeded, >

\newtheorem{theorem}{Theorem}[section]
\newtheorem{problem}{Problem}
\newtheorem{proposition}{Proposition}[section]
\newtheorem{lemma}{Lemma}[section]
\newtheorem{corollary}[theorem]{Corollary}                                                           \newtheorem{example}{Example}[section]
\newtheorem{definition}[problem]{Definition}
%\newtheorem{thm}{Theorem}[section]
%\newtheorem{defn}[thm]{Definition}
%\newtheorem{algorithm}{Algorithm}[section]
%\newtheorem{cor}{Corollary}
\newcommand{\BEQA}{\begin{eqnarray}}
\newcommand{\EEQA}{\end{eqnarray}}
%\newcommand{\define}{\stackrel{\triangle}{=}}
\theoremstyle{remark}
\newtheorem{rem}{Remark}

%\bibliographystyle{ieeetr}

\begin{document}
\title{LATEX ASSIGNMENT}
\author{ANAND}
\date{6-09-2023}
\maketitle
\section*{EXERCISE 12.4.6}
Examine the consistency of the system of equations in  $\ref{prob:1}$ to $\ref{prob:6}$
\begin{enumerate}
\item  \label{prob:1}
\begin{align}
 x+2y = 2\\
2x+3y = 3
\end{align}
\item
\begin{align}
 2x-y = 5\\
 x+y = 4
\end{align}
\item
\begin{align}
 x+3y = 5\\
 2x+6y= 8
\end{align}
\item
\begin{align}
 x+y+z = 1\\
 2x+3y+2z = 2\\
 ax+ay+2az = 4
\end{align}
\item
\begin{align}
 3x-y-2z = 2\\
 2y-z = -1\\
 3x-5y = 3
\end{align}
\item  \label{prob:6}
\begin{align}
 5x-y+4z = 5\\
 2x+3y+5z = 2\\
 5x-2y+6z = -1
\end{align}
\end{enumerate}
 Solve system of linear equations, using matrix method, in $\ref{prob:7}$ to $\ref{prob:14}$.
\begin{enumerate}[resume]
\item   \label{prob:7}
\begin{align}
5x+2y = 4\\
7x+3y = 5
\end{align}
\item 
\begin{align}
 2x-y = -2\\
 3x+4y = 3
\end{align}
\item
\begin{align}
 4x-3y = 3\\
 3x-5y = 7
\end{align}
\item
\begin{align} 
 5x+2y = 3\\
 3x+2y = 5
\end{align}
\item
\begin{align}
 2x+y+z = 1\\
 x-2y-z = \frac{3}{2} \\
 3y-5z = 9
\end{align}
\item
\begin{align}
 x-y+z = 4\\
 2x+y-3z = 0\\
 x+y+z = 2
\end{align}
\item
\begin{align}
 2x+3y+3z = 5\\
 x-2y+z = -4\\
 3x-y-2z = 3
\end{align}
\item   \label{prob:14}
\begin{align}
 x-y+2z = 7\\
 3x+4y-5z = -5\\
 2x-y+3z = 12
\end{align}
\item If $A=\myvec{2&-3&5\\ 3&2&-4\\ 1&1&-2}$, find $A^{-1}$. Using $A^{-1}$ solve the system of equations
\begin{align}
 2x-3y+5z =  11\\
 3x+2y-4z = -5\\
 x+y-2z = -3
\end{align}
\item The cost of $4 kg$ onion, $3 kg$ wheat and $2 kg$ rice is \rupee~60. The cost of $2 kg$ onion, $4 kg$ wheat and $6 kg$ rice is \rupee~90. The cost of $6 kg$ onion, $2 kg$ wheat and $3 kg$ rice is \rupee~70. Find cost of each item per kg by matrix method.
\end{enumerate}
\end{document}
