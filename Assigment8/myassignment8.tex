\let\negmedspace\undefined
\let\negthickspace\undefined
\documentclass{article}
\usepackage{cite}
\usepackage{amsmath,amssymb,amsfonts,amsthm}
\usepackage{algorithmic}
\usepackage{graphicx}
\usepackage{textcomp}
\usepackage{xcolor}
\usepackage{txfonts}
\usepackage{listings}
\usepackage{enumitem}
\usepackage{mathtools}
\usepackage{gensymb}
\usepackage{tfrupee}
\usepackage[breaklinks=true]{hyperref}
\usepackage{tkz-euclide} % loads  TikZ and tkz-base
\usepackage{listings}
\usepackage{gvv}
%
%\usepackage{setspace}
%\usepackage{gensymb}
%\doublespacing
%\singlespacing

%\usepackage{graphicx}
%\usepackage{amssymb}
%\usepackage{relsize}
%\usepackage[cmex10]{amsmath}
%\usepackage{amsthm}
%\interdisplaylinepenalty=2500
%\savesymbol{iint}
%\usepackage{txfonts}
%\restoresymbol{TXF}{iint}
%\usepackage{wasysym}
%\usepackage{amsthm}
%\usepackage{iithtlc}
%\usepackage{mathrsfs}
%\usepackage{txfonts}
%\usepackage{stfloats}
%\usepackage{bm}
%\usepackage{cite}
%\usepackage{cases}
%\usepackage{subfig}
%\usepackage{xtab}
%\usepackage{longtable}
%\usepackage{multirow}
%\usepackage{algorithm}
%\usepackage{algpseudocode}
%\usepackage{enumitem}
%\usepackage{mathtools}
%\usepackage{tikz}
%\usepackage{circuitikz}
%\usepackage{verbatim}
%\usepackage{tfrupee}
%\usepackage{stmaryrd}
%\usetkzobj{all}
%    \usepackage{color}                                            %%
%    \usepackage{array}                                            %%
%    \usepackage{longtable}                                        %%
%    \usepackage{calc}                                             %%
%    \usepackage{multirow}                                         %%
%    \usepackage{hhline}                                           %%
%    \usepackage{ifthen}                                           %%
  %optionally (for landscape tables embedded in another document): %%
%    \usepackage{lscape}
%\usepackage{multicol}
%\usepackage{chngcntr}
%\usepackage{enumerate}

%\usepackage{wasysym}
%\documentclass[conference]{IEEEtran}
%\IEEEoverridecommandlockouts
% The preceding line is only needed to identify funding in the first footnote. If that is unneeded, please comment it out.

\newtheorem{theorem}{Theorem}[section]
\newtheorem{problem}{Problem}
\newtheorem{proposition}{Proposition}[section]
\newtheorem{lemma}{Lemma}[section]
\newtheorem{corollary}[theorem]{Corollary}
\newtheorem{example}{Example}[section]
\newtheorem{definition}[problem]{Definition}
%\newtheorem{thm}{Theorem}[section]
%\newtheorem{defn}[thm]{Definition}
%\newtheorem{algorithm}{Algorithm}[section]
%\newtheorem{cor}{Corollary}
\newcommand{\BEQA}{\begin{eqnarray}}
\newcommand{\EEQA}{\end{eqnarray}}
%\newcommand{\define}{\stackrel{\triangle}{=}}
\theoremstyle{remark}
\newtheorem{rem}{Remark}

%\bibliographystyle{ieeetr}
\begin{document}
\title{LATEX ASSIGNMENT}
\author{ANAND}
\date{25-08-2023}
\maketitle
\section*{EXERCISE 10.7.1}
\begin{enumerate}
\item Find the distance between the following pairs of points:
\begin{enumerate}[label=(\roman*)]
\item $(2,3), (4,1)$
\item $(-5,7), (-1,3)$
\item $(a,b), (-a,b)$
\end{enumerate}
\item Find the distance between the points $(0,0)$ and $(36,15)$. Can you now find the two town A and B discussed in section $7.2$.
\item Determine if the points $(1,5), (2,3)$ and $(-2,11)$ are collinear.
\item Check whether $(5,2), (6,4)$ and $(7,2)$ are the vertices of an isoceles triangle.
\item In a classroom, $4$ friends are seated at the points $A, B, C$ and $D$ as shown \figref{fig:7.8} in  Champa and chameli walk into the class and  after observing for a fwe minutes champa asks chameli, \textquotedblleft Don't you think ABCD is a square?\textquotedblright  Chameli disagrees Using distance formula, find which of them is correct.
	\begin{figure}[ht]
\centering
\includegraphics[width=\columnwidth]{figs/7.8.png}
\caption{7.8}
  \label{fig:7.8}
\end{figure}
\item Name the type of quadrilateral formed, if any, by the following points, and give reasons for your answer:
\begin{enumerate}[label=(\roman*)]
\item $(-1,2), (1,0), (-1,2), (3,0)$
\item $(-3,5), (3,1), (0,3), (-1,-4)$
\item $(4,5), (7,6), (4,3), (1,2)$
\end{enumerate}
\item Find the point on the $x$ axis which is equidistant from $(2,5)$ and $(2,9)$. 
\item Find the values of $y$ for which the distance between the points $P(2,-3)$ and $Q(10,y)$ is $10$ units.
\item $Q(0,1)$ is equidistant from $P(5,-3)$ and $R(x,6)$,find the values of $x$. Also find the distances $QR$ and $PR$.
\item Find a relation  between $x$ and $y$ such that $(x,y)$ is equidistant from the point $(3,6)$ and $(-3,4)$.
\end{enumerate}
\end{document}
	

