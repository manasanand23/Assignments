\let\negmedspace\undefined
\let\negthickspace\undefined
\documentclass{article}
\usepackage{cite}
\usepackage{amsmath,amssymb,amsfonts,amsthm}
\usepackage{algorithmic}
\usepackage{graphicx}
\usepackage{textcomp}
\usepackage{xcolor}
\usepackage{txfonts}
\usepackage{listings}
\usepackage{enumitem}
\usepackage{tfrupee}
\usepackage{mathtools}
\usepackage{gensymb}
\usepackage{tfrupee}
\usepackage[breaklinks=true]{hyperref}
\usepackage{tkz-euclide} % loads  TikZ and tkz-base
\usepackage{listings}
%\usepackage{gvv}
%
%\usepackage{setspace}
%\usepackage{gensymb}
%\doublespacing
%\singlespacing

%\usepackage{graphicx}
%\usepackage{amssymb}
%\usepackage{relsize}
%\usepackage[cmex10]{amsmath}
%\usepackage{amsthm}
%\interdisplaylinepenalty=2500
%\savesymbol{iint}
%\usepackage{txfonts}
%\restoresymbol{TXF}{iint}
%\usepackage{wasysym}
%\usepackage{amsthm}
%\usepackage{iithtlc}
%\usepackage{mathrsfs}
%\usepackage{txfonts}
%\usepackage{stfloats}
%\usepackage{bm}
%\usepackage{cite}
%\usepackage{cases}
%\usepackage{subfig}
%\usepackage{xtab}
%\usepackage{longtable}
%\usepackage{multirow}
%\usepackage{algorithm}
%\usepackage{algpseudocode}
%\usepackage{enumitem}
%\usepackage{mathtools}
%\usepackage{tikz}
%\usepackage{circuitikz}
%\usepackage{verbatim}
%\usepackage{tfrupee}
%\usepackage{stmaryrd}
%\usetkzobj{all}
%    \usepackage{color}                                            %%
%    \usepackage{array}                                            %%
%    \usepackage{longtable}                                        %%
%    \usepackage{calc}                                             %%
%    \usepackage{multirow}                                         %%
%    \usepackage{hhline}                                           %%
%    \usepackage{ifthen}                                           %%
  %optionally (for landscape tables embedded in another document): %%
%    \usepackage{lscape}     
%\usepackage{multicol}
%\usepackage{chngcntr}
%\usepackage{enumerate}

%\usepackage{wasysym}
%\documentclass[conference]{IEEEtran}
%\IEEEoverridecommandlockouts
% The preceding line is only needed to identify funding in the first footnote. If that is unneeded, please comment it out.

\newtheorem{theorem}{Theorem}[section]
\newtheorem{problem}{Problem}
\newtheorem{proposition}{Proposition}[section]
\newtheorem{lemma}{Lemma}[section]
\newtheorem{corollary}[theorem]{Corollary}
\newtheorem{example}{Example}[section]
\newtheorem{definition}[problem]{Definition}
%\newtheorem{thm}{Theorem}[section] 
%\newtheorem{defn}[thm]{Definition}
%\newtheorem{algorithm}{Algorithm}[section]
%\newtheorem{cor}{Corollary}
\newcommand{\BEQA}{\begin{eqnarray}}
\newcommand{\EEQA}{\end{eqnarray}}
\newcommand{\define}{\stackrel{\triangle}{=}}
\theoremstyle{remark}
\newtheorem{rem}{Remark}

%\bibliographystyle{ieeetr}

\begin{document}
\title{LATEX ASSIGNMENT}
\author{ANAND}
\date{28-08-2023}
\maketitle
\section*{EXERCISE 11.10.2}
 In excercises $1$ to $8$, find the equation of the line which satisfy
 the given conditions:
\begin{enumerate}
\item Write the equations for $x$ and $y$ axes.
\item Passing through the point $(-4,3)$ with slope $\frac {1}{2}$.
\item Passing through $(0,0)$ with slope $m$.
\item Passing through $(2,\sqrt {3})$ and inclined with $x$ axis at an angle of $75\degree$.
\item Intersecting the $x$ axis at a distance of $3$ units to the left of the origin with slope $-2$.
\item Intersecting the $y$ axis at a distance of $2$ units above the origin and making an angle of $30°$ with positive direction of the $x$ axis.
\item Passing through the points $(-1,1)$ and $(2,-4)$.
\item Perpendicular distance from the origin is $5$ units and the angle made by the perpendicular with the positive $x$ axis is $30°$. 
\item The vertices of $\triangle PQR$ are $P(2,1), Q(-2,3)$ and $R(4,5)$. Find equation of the median through the vertex $R$.
\item Find the equation of the line passing through $(-3,5)$ and perpendicular to the line through the points $(2,5)$ and $(-3,6)$.
\item A line perpendicular to the line segment joining the points $(1,0)$  and $(2,3)$ divides it in the ratio $1:n$. Find the equation of the line.
\item Find the equation of the line that cuts off equal axes and passes through the point $(2,3)$.
\item Find equation of the line passing through the point $(2,2)$ and cutting off intercepts on the axes whose sum is $9$.
\item Find equation of the line through the point $(0,2)$ making an angle $\frac{2\pi}{3}$ with the positive $x$ axis. Also, find the equation of the parallel to it and crossing the $y$ axis at a distance of $2$ units below the origin.
\item The perpendicular from the origin to a line meets it at the point $(-2,9)$, find the equation of the line.
\item The length $L$ [in centimetre of a copper rod is a linear function of its celsius temperature $C$]. In an experiment, if $L=124.942$. When $C=20$  and $L=125.134$ When $C=110$, express $L$ in terms of $C$.
\item The owner of a milk store finds that, he can sell $980$ litres of milk each week at \rupee~14/litre and $1220$ litres of milk each week at \rupee~16/litre. Assuming a linear relationship between selling price and demand, how many litres could he sell weekly at \rupee~17/ litre?
\item $P(a,b)$ is the mid-point of a line segment between axes. Show that equation of the line is $\frac{x}{a}+\frac{y}{b}=2$
\item Point $R(h,k)$ divides a line segment betwween the axes in the ratio $1:2$. find equation of the line.
\item By Using the concept of equation of a line, prove that the three points (3,0), (-2,-2) and (8,2) are collinear.
\end{enumerate}
\end{document}

