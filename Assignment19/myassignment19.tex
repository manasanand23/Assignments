\let\negmedspace\undefined
\let\negthickspace\undefined
\documentclass{article}
\usepackage{cite}
\usepackage{amsmath,amssymb,amsfonts,amsthm}
\usepackage{algorithmic}
\usepackage{graphicx}
\usepackage{textcomp}
\usepackage{xcolor}
\usepackage{txfonts}
\usepackage{listings}
\usepackage{enumitem}
\usepackage{tfrupee}
\usepackage{mathtools}
\usepackage{gensymb}
\usepackage{tfrupee}
\usepackage[breaklinks=true]{hyperref}
\usepackage{tkz-euclide} % loads  TikZ and tkz-base
\usepackage{listings}
\usepackage{gvv}
%
%\usepackage{setspace}
%\usepackage{gensymb}
%\doublespacing
%\singlespacing

%\usepackage{graphicx}                                                                              >
%\usepackage{relsize}
%\usepackage[cmex10]{amsmath}                                                                        %\usepackage{amsthm}                                                       >
%\savesymbol{iint}                                                                                   %\usepackage{txfonts}
%\restoresymbol{TXF}{iint}                                                                           %\usepackage{wasysym}
%\usepackage{amsthm}                                                                                 %\usepackage{iithtlc}
%\usepackage{mathrsfs}                                                                               %\usepackage{txfonts}
%\usepackage{stfloats}
%\usepackage{bm}
%\usepackage{cite}
%\usepackage{cases}
%\usepackage{subfig}
%\usepackage{xtab}
%\usepackage{longtable}
%\usepackage{multirow}
%\usepackage{algorithm}
%\usepackage{algpseudocode}
%\usepackage{enumitem}                                                                              >
%\usepackage{circuitikz}
%\usepackage{verbatim}
%\usepackage{tfrupee}
%\usepackage{stmaryrd}
%\usetkzobj{all}
%    \usepackage{color}                                            %%
%    \usepackage{array}                                            %%
%    \usepackage{longtable}                                        %%
%    \usepackage{calc}                                             %%
%    \usepackage{multirow}                                         %%
%    \usepackage{hhline}                                           %%
%    \usepackage{ifthen}                                           %%
  %optionally (for landscape tables embedded in another document): %%
%    \usepackage{lscape}
%\usepackage{multicol}
%\usepackage{chngcntr}
%\usepackage{enumerate}

%\usepackage{wasysym}                                                                               >
%\IEEEoverridecommandlockouts
% The preceding line is only needed to identify funding in the first footnote. If that is unneeded, >

\newtheorem{theorem}{Theorem}[section]
\newtheorem{problem}{Problem}
\newtheorem{proposition}{Proposition}[section]
\newtheorem{lemma}{Lemma}[section]
%\newtheorem{corollary}[theorem]{Corollary}                                                          >
\newtheorem{definition}[problem]{Definition}
%\newtheorem{thm}{Theorem}[section]
%\newtheorem{defn}[thm]{Definition}
%\newtheorem{algorithm}{Algorithm}[section]
%\newtheorem{cor}{Corollary}
\newcommand{\BEQA}{\begin{eqnarray}}
\newcommand{\EEQA}{\end{eqnarray}}
%\newcommand{\define}{\stackrel{\triangle}{=}}
\theoremstyle{remark}
\newtheorem{rem}{Remark}

%\bibliographystyle{ieeetr}

\begin{document}
\title{LATEX ASSIGNMENT}
\author{ANAND}
\date{06-09-2023}
\maketitle                                                                       >
\section*{EXERCISE 12.10.2}
\begin{enumerate}
\item Compute the magnitude of the following vectors:
\begin{align*}
\overrightarrow{a}=\hat{i}+\hat{j}+k; \overrightarrow{b}=2\hat{i}-7\hat{j}-3\hat{k}; \overrightarrow{c}=\frac{1}{\sqrt{3}}\hat{i}+\frac{1}{\sqrt{3}}\hat{j}-\frac{1}{3}\hat{k}
\end{align*}
\item Write two different vectors having same magnitude.
\item Write two different vectors having same direction.
\item Find the values of $x$ and $y$ so that the vectors $2\hat{i}+3\hat{j}$ and $x\hat{i}+y\hat{j}$ are equal.
\item Find the scalar and vector components of the vector with initial point $(2, 1)$ and terminal point $(– 5, 7)$.
\item Find the sum of the vectors $\overrightarrow{a}=\hat{i}-2\hat{j}+\hat{k}$, $\overrightarrow{b}=-2\hat{i}+4\hat{j}+5\hat{k}$ and $\overrightarrow{c}=\hat{i}-6\hat{j}-7\hat{k}$.
\item Find the unit vector in the direction of the vector $\overrightarrow{a}=\hat{i}+\hat{j}+2\hat{k}$.
\item Find the unit vector in the direction of vector $\overrightarrow{PQ}$, where $P$ and $Q$ are the points $(1, 2, 3)$ and $(4, 5, 6)$, respectively.
\item For given vectors, $\overrightarrow{a}=2\hat{i}-\hat{j}+2\hat{k}$ and $\overrightarrow{b}=-\hat{i}+\hat{j}-\hat{k}$, find the unit vector in the direction of the vector $\overrightarrow{a}+\overrightarrow{b}$.
\item Find a vector in the direction of vector $5\hat{i}-\hat{j}+2\hat{k}$ which has magnitude 8 units.
\item Show that the vectors $2\hat{i}-3\hat{j}+4\hat{k}$ and $-4\hat{i}+6\hat{j}-8\hat{k}$ are collinear.
\item Find the direction cosines of the vector $\hat{i}+2\hat{j}+3\hat{k}$.
\item Find the direction cosines of the vector joining the points $A(1, 2, –3)$ and $B(–1, –2, 1)$, directed from $A$ to $B$.
\item Show that the vector $\hat{i}+\hat{j}+\hat{k}$ is equally inclined to the axes $OX, OY$ and $OZ$.
\item Find the position vector of a point $R$ which divides the line joining two points $P$ and $Q$ whose position vectors are $\hat{i}+2\hat{j}-\hat{k}$ and $-\hat{i}+\hat{j}+\hat{k}$ respectively, in the ratio 2 : 1
\begin{enumerate}[label=(\roman*)]
    \item  internally
    \item  externally
\end{enumerate}
\item Find the position vector of the mid point of the vector joining the points $P(2, 3, 4)$ and $Q(4, 1, –2)$.
\item Show that the points $A, B$ and $C$ with position vectors, $\overrightarrow{a}=3\hat{i}-4\hat{j}-4\hat{k}, \overrightarrow{b}=2\hat{i}-\hat{j}+\hat{k}$ and $\overrightarrow{c}=\hat{i}-3\hat{j}-5\hat{k}$, respectively form the vertices of a right angled triangle.
\item In triangle $ABC$ \figref{fig:10.8}, which of the following is not true:
 \begin{enumerate}
         \item $\overrightarrow{AB}+\overrightarrow{BC}+\overrightarrow{CA}=0$
         \item $\overrightarrow{AB}+\overrightarrow{BC}-\overrightarrow{CA}=0$
         \item $\overrightarrow{AB}+\overrightarrow{BC}-\overrightarrow{CA}=0$
         \item $\overrightarrow{AB}-\overrightarrow{BC}+\overrightarrow{CA}=0$
\end{enumerate}
\begin{figure}[!h]
\centering
  \includegraphics[width=\columnwidth]{figs/10.18.png}
\caption{10.18}
\label{fig:10.18}
\end{figure}
\item If $a$ and $b$ are two collinear vectors, then which of the following are incorrect:
\begin{enumerate}
    \item $\overrightarrow{b}=\lambda \overrightarrow{a}$, for some scalar $\lambda$
    \item $\overrightarrow{a}=\pm \overrightarrow{b}$
    \item The respective components of $\overrightarrow{a}$ and $\overrightarrow{b}$ are not proportional
    \item Both the vectors $\overrightarrow{a}$ and $\overrightarrow{b}$ have same direction, but different magnitudes.
\end{enumerate}
\end{enumerate}
\end{document}
