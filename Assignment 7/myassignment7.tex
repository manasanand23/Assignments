\let\negmedspace\undefined
\let\negthickspace\undefined
\documentclass{article}
\usepackage{cite}
\usepackage{amsmath,amssymb,amsfonts,amsthm}
\usepackage{algorithmic}
\usepackage{graphicx}
\usepackage{textcomp}
\usepackage{xcolor}
\usepackage{txfonts}
\usepackage{listings}
\usepackage{enumitem}
\usepackage{mathtools}
\usepackage{gensymb}
\usepackage{tfrupee}
\usepackage[breaklinks=true]{hyperref}
\usepackage{tkz-euclide} % loads  TikZ and tkz-base
\usepackage{listings}
%\usepackage{gvv}
%
%\usepackage{setspace}
%\usepackage{gensymb}
%\doublespacing
%\singlespacing

%\usepackage{graphicx}
%\usepackage{amssymb}
%\usepackage{relsize}
%\usepackage[cmex10]{amsmath}
%\usepackage{amsthm}
%\interdisplaylinepenalty=2500
%\savesymbol{iint}
%\usepackage{txfonts}
%\restoresymbol{TXF}{iint}
%\usepackage{wasysym}
%\usepackage{amsthm}
%\usepackage{iithtlc}
%\usepackage{mathrsfs}
%\usepackage{txfonts}
%\usepackage{stfloats}
%\usepackage{bm}
%\usepackage{cite}
%\usepackage{cases}
%\usepackage{subfig}
%\usepackage{xtab}
%\usepackage{longtable}
%\usepackage{multirow}
%\usepackage{algorithm}
%\usepackage{algpseudocode}
%\usepackage{enumitem}
%\usepackage{mathtools}
%\usepackage{tikz}
%\usepackage{circuitikz}
%\usepackage{verbatim}
%\usepackage{tfrupee}
%\usepackage{stmaryrd}
%\usetkzobj{all}
%    \usepackage{color}                                            %%
%    \usepackage{array}                                            %%
%    \usepackage{longtable}                                        %%
%    \usepackage{calc}                                             %%
%    \usepackage{multirow}                                         %%
%    \usepackage{hhline}                                           %%
%    \usepackage{ifthen}                                           %%
  %optionally (for landscape tables embedded in another document): %%
%    \usepackage{lscape}     
%\usepackage{multicol}
%\usepackage{chngcntr}
%\usepackage{enumerate}

%\usepackage{wasysym}
%\documentclass[conference]{IEEEtran}
%\IEEEoverridecommandlockouts
% The preceding line is only needed to identify funding in the first footnote. If that is unneeded, please comment it out.

\newtheorem{theorem}{Theorem}[section]
\newtheorem{problem}{Problem}
\newtheorem{proposition}{Proposition}[section]
\newtheorem{lemma}{Lemma}[section]
\newtheorem{corollary}[theorem]{Corollary}
\newtheorem{example}{Example}[section]
\newtheorem{definition}[problem]{Definition}
%\newtheorem{thm}{Theorem}[section] 
%\newtheorem{defn}[thm]{Definition}
%\newtheorem{algorithm}{Algorithm}[section]
%\newtheorem{cor}{Corollary}
\newcommand{\BEQA}{\begin{eqnarray}}
\newcommand{\EEQA}{\end{eqnarray}}
\newcommand{\define}{\stackrel{\triangle}{=}}
\theoremstyle{remark}
\newtheorem{rem}{Remark}

%\bibliographystyle{ieeetr}
\begin{document}
\title{LATEX ASSIGNMENT}
\author{ANAND}
\date{24-08-2023}
\maketitle
\section*{EXERCISE 10.4.3}
\begin{enumerate}
\item Find the roots of the following quadratic equations, if they exist, by the method of completing the square:
\begin{enumerate}[label=(\roman*)]
\item
$2x^2-7x+3=0$
\item
$2x^2+x-4=0$
\item
$4x^2+4\sqrt 3x+3=0$
\item
$2x^2+x+4=0$
\end{enumerate}
\item Find the roots of the quadratic equations given in Q.1 above by applying the quadratic formula.
\item Find the roots of the following equations:
\begin{enumerate}[label=(\roman*)]
\item
\begin{align}
x-\frac{1}{x}=3, x\neq{0}
\end{align}
\item
\begin{align}
\frac{1}{x+4}-\frac{1}{x-7}=\frac{11}{30}, x\neq{-4,7}
\end{align}
\end{enumerate}
\item The sum of the reciprocals of Rehman's ages, (in years) 3 years ago and 5 years from now is $\frac{1}{3}$. Find his present age.
\item In a class test, the sum of shefali's  marks in Mathematics and english is $30$. Had she got $2$ marks more in Mathematics and $3$ marks less in English, the product of their marks would have been $210$. Find her marks in the two subjects. 
\item The diagonal of a rectangular field is $60$ metres more than the Shorter side. If the longer side is $30$ metres more than the shorter side, find the sides of the field.
\item The difference of squares of two numbers is $180$. The square of the smaller number is $8$ times the larger number. Find the two numbers.
\item A train travels $360$ km at a uniform speed. If the speed had been $5$ km/hr more, it would have taken $1$ hour less for the same journey. Find the speed of the train.
\item Two Water taps together can fill a tank in $9\frac{3}{8}$ hours.The tap of larger diameter takes $10$ hours.The tap of larger diameter takes $10$ hours less than the smaller one to fill the tank seperately. Find the time in which each tap can seperately fill the  tank.
\item An express train takes $1$ hour less than a passenger train to travel 132 km between mysore and bangalore (without taking into consideration the time they stop at intermediate statioons). If the average speed of the express train is 11 Km/h more than that of the passenger train, find the average speed of the two trains.
\item Sum of the areas of two square is $468m^2$. If the difference of their perimeter is 24m, find the sides of the two squares.  
\end{enumerate}
\end{document}

