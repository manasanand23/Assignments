\let\negmedspace\undefined
\let\negthickspace\undefined
\documentclass{article}
\usepackage{cite}
\usepackage{amsmath,amssymb,amsfonts,amsthm}
\usepackage{algorithmic}
\usepackage{graphicx}
\usepackage{textcomp}
\usepackage{xcolor}
\usepackage{txfonts}
\usepackage{listings}
\usepackage{enumitem}
\usepackage{tfrupee}
\usepackage{mathtools}
\usepackage{gensymb}
\usepackage{tfrupee}
\usepackage[breaklinks=true]{hyperref}
\usepackage{tkz-euclide} % loads  TikZ and tkz-base
\usepackage{listings}
\usepackage{gvv}
%
%\usepackage{setspace}
%\usepackage{gensymb}
%\doublespacing
%\singlespacing

%\usepackage{graphicx}                                                                              >
%\usepackage{relsize}
%\usepackage[cmex10]{amsmath}                                                                       >
%\savesymbol{iint}                                                                                  >
%\restoresymbol{TXF}{iint}                                                                          >
%\usepackage{amsthm}                                                                                >
%\usepackage{mathrsfs}                                                                              >
%\usepackage{stfloats}
%\usepackage{bm}
%\usepackage{cite}
%\usepackage{cases}
%\usepackage{subfig}
%\usepackage{xtab}
%\usepackage{longtable}
%\usepackage{multirow}
%\usepackage{algorithm}
%\usepackage{algpseudocode}
%\usepackage{enumitem}                                                                              >
%\usepackage{circuitikz}
%\usepackage{verbatim}
%\usepackage{tfrupee}
%\usepackage{stmaryrd}
%\usetkzobj{all}
%    \usepackage{color}                                            %%
%    \usepackage{array}                                            %%
%    \usepackage{longtable}                                        %%
%    \usepackage{calc}                                             %%
%    \usepackage{multirow}                                         %%
%    \usepackage{hhline}                                           %%
%    \usepackage{ifthen}                                           %%
  %optionally (for landscape tables embedded in another document): %%
%    \usepackage{lscape}
%\usepackage{multicol}
%\usepackage{chngcntr}
%\usepackage{enumerate}

%\usepackage{wasysym}                                                                               >
%\IEEEoverridecommandlockouts
% The preceding line is only needed to identify funding in the first footnote. If that is unneeded, >

\newtheorem{theorem}{Theorem}[section]
\newtheorem{problem}{Problem}
\newtheorem{proposition}{Proposition}[section]
\newtheorem{lemma}{Lemma}[section]
%\newtheorem{corollary}[theorem]{Corollary}                                                         >
\newtheorem{definition}[problem]{Definition}
%\newtheorem{thm}{Theorem}[section]
%\newtheorem{defn}[thm]{Definition}
%\newtheorem{algorithm}{Algorithm}[section]
%\newtheorem{cor}{Corollary}
\newcommand{\BEQA}{\begin{eqnarray}}
\newcommand{\EEQA}{\end{eqnarray}}
%\newcommand{\define}{\stackrel{\triangle}{=}}
\theoremstyle{remark}
\newtheorem{rem}{Remark}

%\bibliographystyle{ieeetr}

\begin{document}
\title{LATEX ASSIGNMENT}
\author{ANAND}
\date{9-09-2023}
\maketitle
\section*{EXERCISE 12.11.4}
\begin{enumerate}
\item Show that the link joining the origin to the point $(2,1,1)$ is perpendicular to the point $(2,1,1)$ is perpendicular to the line determined by the points $(2,1,1)$ is perpendicular to the line determined by the points $(3,5,-1)$, $(4,3,-1)$.
\item If $l_1,m_1,n_1$ and $l_2,m_2,n_2$ are the direction cosines of two mutually perpendicular lines, show that the direction cosines of the line perpendicular to Both of these are $m_{1}n_{2}-m_{2}n_{1}, n_{1}l_{2}-n_{2}l_{2}-n_{2}l_{1}, l_{1}m_{2}-l_{2}m_{1}$.
\item Find the angle between the lines whose direction ratios are $a, b, c$ and $b-c$, $c-a$, $a-b$.
\item Find the equation of a line parallel to $x$-axis and passing through the origin.
\item If the co-ordinates of the points $A,B,C,D$ be $(1,2,3)$, $(4,5,7)$, $(-4,3,-6)$ and $(2,9,2)$ respectively, then find the angle between the lines $AB$ and $CD$.
\item If the lines $\frac{x-1}{-3}=\frac{y-2}{2k}=\frac{z-3}{2}$ and $\frac{x-1}{3k}=\frac{y-1}{1}=\frac{z-6}{-5}$ are perpendicular, Find the value of $k$.
\item Find the vector equation of the line passing through $(1,2,3)$ and perpendicular to the plane $\overrightarrow{y}\cdot(\hat{i}+2\hat{j}-5\hat{k})+9=0$ .
\item Find the equation of the plane passing through $(a,b,c)$ and parallel to the plane $\overrightarrow{r}(\hat{i}+\hat{j}+\hat{k})=2$.
\item Find the shortest distance between the lines $\overrightarrow{r}=6\hat{i}+2\hat{j}+2\hat{k}+\lambda(\hat{i}-2\hat{j}+2\hat{k})$ and $\overrightarrow{r}=4\hat{i}-\hat{k}+\mu(3\hat{i}-2\hat{j}-2\hat{k})$.
\item Find the co ordinates of the point where the line through $(5,1,6)$ and $(3,4,1)$ crosses the $YZ$-plane.
\item Find the co ordinates of the point where the line through $(5,1,6)$ and $(3,4,1)$ crosses the $ZX$-plane.
\item Find the co ordinates of the point where the line through $(3,-4,-5)$ and $(2,-3,1)$ crosses the plane $2x+y+x=7$.
\item Find the equation of the plane passing through the point $(-1,3,2)$ and perpendicular to each of the planes $x+2y+3z=5$ and $3x+3y+z=0$.
\item If the points $(1,1,p)$ and $(-3,0,1)$ be equidistant from the plane $\overrightarrow{r}\cdot(3\hat{i}+4\hat{j}-12\hat{k})+13=0$, then find the value of $p$.
\item Find the equation of the plane passing through the line of intersection of the planes $\overrightarrow{r}\cdot(\hat{i}\hat{j}+\hat{k})=1$ and $\overrightarrow{r}\cdot(\hat{i}+\hat{j}+\hat{k})=1$ and $\overrightarrow{r}\cdot(2\hat{i}+3\hat{j}-\hat{k})+4=0$ and parallel to $x$-axis.
\item If $0$ be the origin and the coordinates of $p$ be $(1,2,-3)$, then find the equation of the plane passing through $P$ and perpendicular to $OP$.
\item Find the equation of the plane which contains the line of intersection of the planes $\overrightarrow{r}\cdot(2\hat{i}+\hat{j}-\hat{k})+5=0$ and which is perpendicular to the plane $\overrightarrow{r}\cdot(5\hat{i}+\hat{j}-6\hat{k})+8=0$
\item Find the distance of the point $(-1,-5,-10)$ from the point of intersection of the line $\overrightarrow{r}=2\hat{i}-\hat{j}-2\hat{k}+\lambda(3\hat{i}+4\hat{j}+2\hat{k})$ and the plane $\overrightarrow{r}\cdot(\hat{i}-\hat{j}+\hat{k})=5$.
\item Find the vector equations of the line passing through the point $(1,2,-4)$ and perpendicular to the two lines:
\begin{align}
\frac{x-8}{3}=\frac{y+19}{-16}=\frac{z-10}{7}\text{ and } \frac{x-15}{3}=\frac{y-29}{8}=\frac{z-5}{-5}.
\end{align}
\item Prove that if a plane has the intercept $a,b,c$ and is at a distance of $p$ units  from the origin, then
\begin{align}
\frac{1}{a^2}+\frac{1}{b^2}+\frac{1}{c^2}=\frac{1}{p^2}.
\end{align}
\end{enumerate}
Choose the correct answer in Excercises $22$ and $23$
\begin{enumerate}[resume]
\item Distance between the two planes: $2x+3y+4z=4$ and $4x+6y+8z=12$ is \label{prob:22}
\begin{enumerate}
\item $2$ units
\item $4$ units
\item $8$ units
\item $\frac{2}{\sqrt{29}}$ units
\end{enumerate}
\item The planes: $2x-y+4z=5$ and $5x-2.5y+10z=6$ are \label{prob:23}
\begin{enumerate}
\item Perpendicular
\item Parallel
\item Intersect $y$ axis
\item Passes through $[0,0,\frac{5}{4}]$
\end{enumerate}
\end{enumerate}
\end{document}
