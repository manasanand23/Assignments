\let\negmedspace\undefined
\let\negthickspace\undefined
\documentclass{article}
\usepackage{cite}
\usepackage{amsmath,amssymb,amsfonts,amsthm}
\usepackage{algorithmic}
\usepackage{graphicx}
\usepackage{textcomp}
\usepackage{xcolor}
\usepackage{txfonts}
\usepackage{listings}
\usepackage{enumitem}
\usepackage{tfrupee}
\usepackage{mathtools}
\usepackage{gensymb}
\usepackage{tfrupee}
\usepackage[breaklinks=true]{hyperref}
\usepackage{tkz-euclide} % loads  TikZ and tkz-base
\usepackage{listings}
\usepackage{gvv}
%
%\usepackage{setspace}
%\usepackage{gensymb}
%\doublespacing
%\singlespacing

%\usepackage{graphicx}                                                                              >
%\usepackage{relsize}
%\usepackage[cmex10]{amsmath}                                                                       >
%\savesymbol{iint}                                                                                  >
%\restoresymbol{TXF}{iint}                                                                          >
%\usepackage{amsthm}                                                                                >
%\usepackage{mathrsfs}                                                                              >
%\usepackage{stfloats}
%\usepackage{bm}
%\usepackage{cite}
%\usepackage{cases}
%\usepackage{subfig}
%\usepackage{xtab}
%\usepackage{longtable}
%\usepackage{multirow}
%\usepackage{algorithm}
%\usepackage{algpseudocode}
%\usepackage{enumitem}                                                                              >
%\usepackage{circuitikz}
%\usepackage{verbatim}
%\usepackage{tfrupee}
%\usepackage{stmaryrd}
%\usetkzobj{all}
%    \usepackage{color}                                            %%
%    \usepackage{array}                                            %%
%    \usepackage{longtable}                                        %%
%    \usepackage{calc}                                             %%
%    \usepackage{multirow}                                         %%
%    \usepackage{hhline}                                           %%
%    \usepackage{ifthen}                                           %%
  %optionally (for landscape tables embedded in another document): %%
%    \usepackage{lscape}
%\usepackage{multicol}
%\usepackage{chngcntr}
%\usepackage{enumerate}

%\usepackage{wasysym}                                                                               >
%\IEEEoverridecommandlockouts
% The preceding line is only needed to identify funding in the first footnote. If that is unneeded, >

\newtheorem{theorem}{Theorem}[section]
\newtheorem{problem}{Problem}
\newtheorem{proposition}{Proposition}[section]
\newtheorem{lemma}{Lemma}[section]
%\newtheorem{corollary}[theorem]{Corollary}                                                         >
\newtheorem{definition}[problem]{Definition}
%\newtheorem{thm}{Theorem}[section]
%\newtheorem{defn}[thm]{Definition}
%\newtheorem{algorithm}{Algorithm}[section]
%\newtheorem{cor}{Corollary}
\newcommand{\BEQA}{\begin{eqnarray}}
\newcommand{\EEQA}{\end{eqnarray}}
%\newcommand{\define}{\stackrel{\triangle}{=}}
\theoremstyle{remark}
\newtheorem{rem}{Remark}

%\bibliographystyle{ieeetr}

\begin{document}
\title{LATEX ASSIGNMENT}
\author{ANAND}
\date{6-09-2023}
\maketitle
\section*{EXERCISE 12.10.4}
\begin{enumerate}
\item Find $\abs{\overrightarrow{a}\times\overrightarrow{b}},\text{ if }\overrightarrow{a}=\hat{i}-7\hat{j}+7\hat{k}\text{ and } \overrightarrow{b}=3\hat{i}-2\hat{j}+2\hat{k}$.
\item Find a unit vector perpendicular to each of the vector $\overrightarrow{a}+\overrightarrow{b}\text{ and }\overrightarrow{a}-\overrightarrow{b},\text{ where } \overrightarrow{a}=3\hat{i}+2\hat{j}+2\hat{k}\text{ and } \overrightarrow{b}=\hat{i}+2\hat{j}-2\hat{k}$. 
\item If a unit vector $\overrightarrow{a}$ makes angles $\dfrac{\pi}{3}\text{ with }\hat{i}, \dfrac{\pi}{4}\text{ with }\hat{j}$ and an acute angle $\theta \text{ with }\hat{k},\text{ then find } \theta$ and hence, the components of $\overrightarrow{a}$.
\item Show that $(\overrightarrow{a}-\overrightarrow{b})\times (\overrightarrow{a}+\overrightarrow{b})=2(\overrightarrow{a}\times \overrightarrow{b})$
\item Find $\lambda$ and $\mu$ if $(2\hat{i}+6\hat{j}+27\hat{k})\times(\hat{i}+\lambda \hat{j} + \mu \hat{k})=\overrightarrow{0}$.
\item Given that $\overrightarrow{a} \cdot \overrightarrow{b} = 0$ and $\overrightarrow{a} \times \overrightarrow{b} = \overrightarrow{0}$. What can you conclude about the vectors $\overrightarrow{a} \text{ and }\overrightarrow{b}$?
\item Let the vectors be given as $\overrightarrow{a},\overrightarrow{b},\overrightarrow{c}\text{ be given as }\ a_1 \hat{i}+\ a_2 \hat{j}+\ a_3 \hat{k},\ b_1 \hat{i}+\ b_2 \hat{j}+\ b_3 \hat{k},\ c_1 \hat{i}+\ c_2 \hat{j}+\ c_3 \hat{k}$. Then show that $\overrightarrow{a} \times (\overrightarrow{b} + \overrightarrow{c}) = \overrightarrow{a} \times \overrightarrow{b}+\overrightarrow{a} \times \overrightarrow{c}$.
\item If either $\overrightarrow{a} = \overrightarrow{0}$ or $\overrightarrow{b} = \overrightarrow{0}$, then $\overrightarrow{a} \times \overrightarrow{b} = \overrightarrow{0}$. Is the converse true? Justify your answer with an example.
\item Find the area of the triangle with vertices $A(1, 1, 2)$, $B(2, 3, 5)$, and $C(1, 5, 5)$
\item Find the area of the parallelogram whose adjacent sides are determined by the vectors $\overrightarrow{a}=\hat{i}-\hat{j}+3\hat{k}$ and $\overrightarrow{b}=2\hat{i}-7\hat{j}+\hat{k}$.
\item Let the vectors $\overrightarrow{a}$ and $\overrightarrow{b}$ be such that $|\overrightarrow{a}| = 3$ and $|\overrightarrow{b}| = \dfrac{\sqrt{2}}{3}$, then $\overrightarrow{a} \times \overrightarrow{b}$ is a unit vector, if the angle between $\overrightarrow{a}$ and $\overrightarrow{b}$ is
\begin{enumerate}
\item $\frac{\pi}{6}$
\item $\frac{\pi}{4}$
\item $\frac{\pi}{3}$
\item $\frac{\pi}{2}$
\end{enumerate}
\item Area of a rectangle having vertices $A, B, C$ and $D$ with position vectors $ -\hat{i}+ \frac{1}{2} \hat{j}+4\hat{k},\hat{i}+ \frac{1}{2} \hat{j}+4\hat{k},\hat{i}-\frac{1}{2} \hat{j}+4\hat{k}\text{ and }-\hat{i}- \dfrac{1}{2} \hat{j}+4\hat{k}$, respectively is
\begin{enumerate}
\item $\frac{1}{2}$
\item 1
\item 2
\item 4
\end{enumerate}
\end{enumerate}
\end{document}

