\let\negmedspace\undefined
\let\negthickspace\undefined
\documentclass{article}
\usepackage{cite}
\usepackage{amsmath,amssymb,amsfonts,amsthm}
\usepackage{algorithmic}
\usepackage{graphicx}
\usepackage{textcomp}
\usepackage{xcolor}
\usepackage{txfonts}
\usepackage{listings}
\usepackage{enumitem}
\usepackage{tfrupee}
\usepackage{mathtools}
\usepackage{gensymb}
\usepackage{tfrupee}
\usepackage[breaklinks=true]{hyperref}
\usepackage{tkz-euclide} % loads  TikZ and tkz-base
\usepackage{listings}
%\usepackage{gvv}
%
%\usepackage{setspace}
%\usepackage{gensymb}
%\doublespacing
%\singlespacing

%\usepackage{graphicx}
%\usepackage{amssymb}
%\usepackage{relsize}
%\usepackage[cmex10]{amsmath}
%\usepackage{amsthm}
%\interdisplaylinepenalty=2500
%\savesymbol{iint}
%\usepackage{txfonts}
%\restoresymbol{TXF}{iint}
%\usepackage{wasysym}
%\usepackage{amsthm}
%\usepackage{iithtlc}
%\usepackage{mathrsfs}
%\usepackage{txfonts}
%\usepackage{stfloats}
%\usepackage{bm}
%\usepackage{cite}
%\usepackage{cases}
%\usepackage{subfig}
%\usepackage{xtab}
%\usepackage{longtable}
%\usepackage{multirow}
%\usepackage{algorithm}
%\usepackage{algpseudocode}
%\usepackage{enumitem}
%\usepackage{mathtools}
%\usepackage{tikz}
%\usepackage{circuitikz}
%\usepackage{verbatim}
%\usepackage{tfrupee}
%\usepackage{stmaryrd}
%\usetkzobj{all}
%    \usepackage{color}                                            %%
%    \usepackage{array}                                            %%
%    \usepackage{longtable}                                        %%
%    \usepackage{calc}                                             %%
%    \usepackage{multirow}                                         %%
%    \usepackage{hhline}                                           %%
%    \usepackage{ifthen}                                           %%
  %optionally (for landscape tables embedded in another document): %%
%    \usepackage{lscape}
%\usepackage{multicol}
%\usepackage{chngcntr}
%\usepackage{enumerate}

%\usepackage{wasysym}
%\documentclass[conference]{IEEEtran}
%\IEEEoverridecommandlockouts
% The preceding line is only needed to identify funding in the first footnote. If that is unneeded, please comment it out.

\newtheorem{theorem}{Theorem}[section]
\newtheorem{problem}{Problem}
\newtheorem{proposition}{Proposition}[section]
\newtheorem{lemma}{Lemma}[section]
\newtheorem{corollary}[theorem]{Corollary}
\newtheorem{example}{Example}[section]
\newtheorem{definition}[problem]{Definition}
%\newtheorem{thm}{Theorem}[section]
%\newtheorem{defn}[thm]{Definition}
%\newtheorem{algorithm}{Algorithm}[section]
%\newtheorem{cor}{Corollary}
\newcommand{\BEQA}{\begin{eqnarray}}
\newcommand{\EEQA}{\end{eqnarray}}
\newcommand{\define}{\stackrel{\triangle}{=}}
\theoremstyle{remark}
\newtheorem{rem}{Remark}

%\bibliographystyle{ieeetr}

\begin{document}
\title{LATEX ASSIGNMENT}
\author{ANAND}
\date{29-08-2023}
\maketitle
\section*{EXERCISE 11.10.4}
\begin{enumerate}
\item Find the values of $K$ for which the line $(K-3), x-(4-k^2) y+k^2-7k+6=0$ is
\begin{enumerate} 
\item Parallel to the $x$ axis.
\item Parallel to the $y$ axis.
\item Passing through the origin.
\end{enumerate}
\item Find the values of $\theta$ and $p$, if the equation $x\cos \theta + y\sin \theta = P$ is the normal form of the line $\sqrt 3x+y+2=0$.
\item Find the equations of the lines, which cut-off intercepts on the axes whose sum and product are $1$ and $-6$, respectively.
\item What are the points on the $y$ axis whose distance from the line $\frac{x}{3}+\frac{y}{4}=1$ is $4$ units.
\item Find perpendicular distance from the origin to the line joining the points $(\cos \theta \sin \theta)$ and $(\cos \phi, \sin \phi)$.
\item Find the equation of the line parallel to $y$ axis and drawn through the point of intersection of the lines $x-7y+5=0$ and $3x+y=0$.
\item Find equation of a line drawn perpendicular to the line $\frac{x}{4}+\frac{y}{6}=1$ through the point, where it meets the $y$ axis.
\item Find the area of the triangle formed by the lines $y-x=0, x+y=0$ and $x-k=0$.
\item Find the value of $p$ so that the three lines $3x+y-2=0, Px+2y-3=0$ and $2x-y-3=0, px+2y-3=0$ and $2x-y-3=0$ may intersect at one point.
\item If three lines when equation are $y=m_1x+c_1y=m_2x+c_2$ and $y=m_1x+c_1$ are concurrent, then show that $m_1 (c_2-c_3)+m_2(c_1-c_2)=0$
\item Find the equation of the lines through the point (3,2) which make an angle of $45\degree$ with the line $x-2y = 3$
\item Find the equation of the line passing through the point of intersection of the lines $4x+7y-3=0$ and $2x-3y+1=0$ that has equal interceptson the axes.
\item Show that the equation of the line passing through the origin and making an angle $\theta$ with the line $y=mx+c$ is $\frac {y}{x}=\frac {m\pm \tan}{1\mp \tan}$
\item In What ratio, the line joining $(-1,1)$ and $(5,7)$ is divided by the line $x+y=4$?
\item Find the distance of the line $4x=7y+5=0$ from the point $(1,2)$ along the line $2x-y=0$.
\item Find the direction in which a straight line must be  drawn through the point $(-1,2)$ so that the point of intersection with the line $x+y=4$ may be at a distance of $3$ units from this point.
\item The hypothesis of a right angled triangle has its ends at the points $(1,3)$ and $(-4,1)$. Find an equation of the legs (perpendicular sides of the triangle.
\item Find the image of the point $(3,8)$ with respect to the line $x+3y=7$ assuming the line to be a plane mirror.
\item If the lines $y= 3x+1$ and $2y= x+3$ are equally inclined to the line $y= mx+4$. Find the value of $m$.
\item If sum of the perpendicular distance of a variable point $P(x,y)$ from the lines $x+y-5=0$ and $3x-2y+7=0$ is always $10$. Show that $P$ must move on a line.
\item Find equation of the line which is equidistant from parallel lines $9x+6y=-7$ and $3x+2y+6=0$.
\item A ray of the light passing through the point $(1,2)$ reflects on the $x$ axis at point $A$ and the reflected ray passes through the point $(5,3)$. Find the coordinates of $A$.
\item Prove that the product of the lengths of the perpendiculars drawn from the points $(\sqrt a^2-b^2, 0)$ and $(\sqrt a^2-b^2, 0)$ to the line $\frac{x}{a} \cos \theta + \frac{y}{b} \sin \theta = lis b^2$
\item A person standing at the junction (crossing) of two straight paths represented by the equations $2x-3y+4=0$ and $3x+4y-5 = 0$ and $3x+4y-5= 0 $ wants to reach the path whose equation is $6x-7y+8=0$ in the least time. Find the equation of the path that he should follow.
\end{enumerate}
\end{document}
