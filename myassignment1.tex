\let\negmedspace\undefined
\let\negthickspace\undefined
%\documentclass[journal,12pt,twocolumn]{IEEEtran}
\documentclass[10pt]{article}
\usepackage{cite}
\usepackage{amsmath,amssymb,amsfonts,amsthm}
\usepackage{algorithmic}
\usepackage{graphicx}
\usepackage{textcomp}
\usepackage{xcolor}
\usepackage{txfonts}
\usepackage{listings}
\usepackage{enumitem}
\usepackage{mathtools}
\usepackage{gensymb}
%\usepackage[breaklinks=true]{hyperref}
%\usepackage{tkz-euclide} % loads  TikZ and tkz-base
%\usepackage{listings}
%\usepackage{gvv}
%\usepackage{setspace}
%\usepackage{gensymb}
%\doublespacing
%\singlespacing
%\usepackage{graphicx}
%\usepackage{amssymb}
%\usepackage{relsize}
%\usepackage[cmex10]{amsmath}
%\usepackage{amsthm}
%\interdisplaylinepenalty=2500
%\savesymbol{iint}
%\usepackage{txfonts}
%\restoresymbol{TXF}{iint}
%\usepackage{wasysym}
%\usepackage{amsthm}
%\usepackage{iithtlc}
%\usepackage{mathrsfs}
%\usepackage{txfonts}
%\usepackage{stfloats}
%\usepackage{bm}
%\usepackage{cite}
%\usepackage{cases}
%\usepackage{subfig}
%\usepackage{xtab}
%\usepackage{longtable}
%\usepackage{multirow}
%\usepackage{algorithm}
%\usepackage{algpseudocode}
%\usepackage{enumitem}
%\usepackage{mathtools}
%\usepackage{tikz}
%\usepackage{circuitikz}
%\usepackage{verbatim}
%\usepackage{tfrupee}
%\usepackage{stmaryrd}
%\usetkzobj{all}
%    \usepackage{color}                                            %%
%    \usepackage{array}                                            %%
%    \usepackage{longtable}                                        %%
%    \usepackage{calc}                                             %%
%    \usepackage{multirow}                                         %%
%    \usepackage{hhline}                                           %%
%    \usepackage{ifthen}                                           %%
  %optionally (for landscape tables embedded in another document): %%
%    \usepackage{lscape}
%\usepackage{multicol}
%\usepackage{chngcntr}
%\usepackage{enumerate}
%\usepackage{wasysym}
%\documentclass[conference]{IEEEtran}
%\IEEEoverridecommandlockouts
% The preceding line is only needed to identify funding in the first footnote. If that is unneeded, please comment it out.
%\newtheorem{theorem}{Theorem}[section]
%\newtheorem{problem}{Problem}
%\newtheorem{proposition}{Proposition}[section]
%\newtheorem{lemma}{Lemma}[section]
%\newtheorem{corollary}[theorem]{Corollary}
%\newtheorem{example}{Example}[section]
%\newtheorem{definition}[problem]{Definition}
%\newtheorem{thm}{Theorem}[section]
%\newtheorem{defn}[thm]{Definition}
%\newtheorem{algorithm}{Algorithm}[section]
%\newtheorem{cor}{Corollary}
%\newcommand{\BEQA}{\begin{eqnarray}}
%\newcommand{\EEQA}{\end{eqnarray}}
%\newcommand{\define}{\stackrel{\triangle}{=}}
%\theoremstyle{remark}
%\newtheorem{rem}{Remark}
%\bibliographystyle{ieeetr}
%\begin{document}
%\bibliographystyle{IEEEtran}
%\vspace{3cm}
%\renewcommand{\thefigure}{\theenumi}
%\renewcommand{\thetable}{\theenumi}
%\renewcommand{\theequation}{\theenumi}
%\begin{figure}
%\centering
%\includegraphics[width=\columnwidth]{./figs/tri_sss.pdf}
%\caption{Triangle generated using python}
%\label{fig:tri_sss_py}
%\end{figure}
%\item Fig. \ref{fig:tri_sss_tikz} is generated using
%\begin{lstlisting}
%math/figs/tri_sss_alone.tex
%\end{lstlisting}
%\begin{figure}[!ht]
        %\begin{center}
        %\resizebox{\columnwidth}{!}{\input{./figs/tri_sss.tex}}
        %\end{center}
        %\caption{Triangle generated using \LaTeX Tikz.}
        %\label{fig:tri_sss_tikz}
%\end{figure}
%\end{enumerate}
%\end{document}
\begin{document}
\title{LATEX ASSIGNMENT}
\author{ANAND}
\date{16-08-2023}
\maketitle
\section*{EXERCISE 9.4.1}
%\begin{enumerate}
\begin{enumerate}[label=\arabic*.,ref=\theenumi]
\item The cost of a notebook is twice the cost of a pen.Write a linear
equation in two variables to represent this statement.
(Take the Cost of a notebook to be $x$ and that of a pen to be
$y$).
\item Express the following linear equation in the form $ax+by+c=0$
and indicate the values of $a,b$ and $c$ in each case:
%\begin{enumerate}
\begin{enumerate}[label=(\roman*),ref=\theenumi]
\item $2x+3y=9.3\overline{5}$
\item $x-\frac{y}{5}-10=10$
\item $-2x+3y=6$
\item $ x=3y$
\item $2x=-5y$
\item $3x+2=0$
\item $y-2=0$
\item $5=2x$
\end{enumerate}
\end{enumerate}
\end{document}
