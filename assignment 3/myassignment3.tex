\let\negmedspace\undefined
\let\negthickspace\undefined
\documentclass{article}
\usepackage{cite}
\usepackage{amsmath,amssymb,amsfonts,amsthm}
\usepackage{algorithmic}
\usepackage{graphicx}
\usepackage{textcomp}
\usepackage{xcolor}
\usepackage{txfonts}
\usepackage{listings}
\usepackage{enumitem}
\usepackage{mathtools}
\usepackage{gensymb}
\usepackage{tfrupee}
\usepackage[breaklinks=true]{hyperref}
\usepackage{tkz-euclide} % loads  TikZ and tkz-base
\usepackage{listings}
%\usepackage{gvv}
%
%\usepackage{setspace}
%\usepackage{gensymb}
%\doublespacing
%\singlespacing

%\usepackage{graphicx}
%\usepackage{amssymb}
%\usepackage{relsize}
%\usepackage[cmex10]{amsmath}
%\usepackage{amsthm}
%\interdisplaylinepenalty=2500
%\savesymbol{iint}
%\usepackage{txfonts}
%\restoresymbol{TXF}{iint}
%\usepackage{wasysym}
%\usepackage{amsthm}
%\usepackage{iithtlc}
%\usepackage{mathrsfs}
%\usepackage{txfonts}
%\usepackage{stfloats}
%\usepackage{bm}
%\usepackage{cite}
%\usepackage{cases}
%\usepackage{subfig}
%\usepackage{xtab}
%\usepackage{longtable}
%\usepackage{multirow}
%\usepackage{algorithm}
%\usepackage{algpseudocode}
%\usepackage{enumitem}
%\usepackage{mathtools}
%\usepackage{tikz}
%\usepackage{circuitikz}
%\usepackage{verbatim}
%\usepackage{tfrupee}
%\usepackage{stmaryrd}
%\usetkzobj{all}
%    \usepackage{color}                                            %%
%    \usepackage{array}                                            %%
%    \usepackage{longtable}                                        %%
%    \usepackage{calc}                                             %%
%    \usepackage{multirow}                                         %%
%    \usepackage{hhline}                                           %%
%    \usepackage{ifthen}                                           %%
  %optionally (for landscape tables embedded in another document): %%
%    \usepackage{lscape}     
%\usepackage{multicol}
%\usepackage{chngcntr}
%\usepackage{enumerate}

%\usepackage{wasysym}
%\documentclass[conference]{IEEEtran}
%\IEEEoverridecommandlockouts
% The preceding line is only needed to identify funding in the first footnote. If that is unneeded, please comment it out.

\newtheorem{theorem}{Theorem}[section]
\newtheorem{problem}{Problem}
\newtheorem{proposition}{Proposition}[section]
\newtheorem{lemma}{Lemma}[section]
\newtheorem{corollary}[theorem]{Corollary}
\newtheorem{example}{Example}[section]
\newtheorem{definition}[problem]{Definition}
%\newtheorem{thm}{Theorem}[section] 
%\newtheorem{defn}[thm]{Definition}
%\newtheorem{algorithm}{Algorithm}[section]
%\newtheorem{cor}{Corollary}
\newcommand{\BEQA}{\begin{eqnarray}}
\newcommand{\EEQA}{\end{eqnarray}}
\newcommand{\define}{\stackrel{\triangle}{=}}
\theoremstyle{remark}
\newtheorem{rem}{Remark}

%\bibliographystyle{ieeetr}
\begin{document}
\title{LATEX ASSIGNMENT}
\author{ANAND}
\date{18-08-2023}
\maketitle
\section*{EXERCISE 10.3.3}
\begin{enumerate}
\item Solve the following pair of linear equations by the substitution method.
    \begin{enumerate}[label=(\Roman*)]
    \item $x+y=14$ \\
          $x-y=4$
    \item $s-t=3$
    \item $3x-y=3$ \\
          $9x-3y=9$
    \item $0.2x+0.3y=1.3$ \\
          $0.4x+0.5y=23$
    \item $\sqrt{2x}+\sqrt{3y}=0$ \\
          $\sqrt{3x}-\sqrt{8y}=0$
    \item $\frac{3x}{2}-\frac{5y}{2}=-2$ \\
          $\frac{x}{3}+\frac{y}{2}=\frac{13}{6}$
    \end{enumerate}
\item Solve $2x+3y=11$ and $2x+4y=-24$ and hence find the value of $m$ for which $y=mx+3$
\item Form the pair of linear equations for the following problems and find their solutions by the substitution method
    \begin{enumerate}[label=(\Roman*)]
    \item The difference between two numbers is $26$ and one number is three times the other. Find them.
    \item The larger of two supplementary angles exceeds the smaller by $18$ degrees. Find them.
    \item The coach of a cricket team buys $7$ balls and $6$ balls for \rupee~3800. Later, she buys $3$ bats and $5$ balls for \rupee~1750. Find the cost of each bat and each ball.
    \item The taxi charges in a city consist of a fixed charge together with the charges for the distance covered. For a distance of $10$ km, the charge paid is \rupee~105 and for a distance of $15$ km, the charge paid is \rupee~155. What are the fixed charges and the charge per km? How much does a person have to travel?
    \item A fraction becomes $\frac{9}{11}$ if $2$ is added to both the numerator and the denominator. If $3$ is added to the numerator and the denominator, it becomes $\frac{5}{6}$. Find the fraction.
    \item Five years hence, the age of Jacob will be three times that of his son. Five years ago, Jacob's age was seven times that of his son. What are their present ages?
    \end{enumerate}
\end{enumerate}
\end{document}
