\let\negmedspace\undefined
\let\negthickspace\undefined
\documentclass{article}
\usepackage{cite}
\usepackage{amsmath,amssymb,amsfonts,amsthm}
\usepackage{algorithmic}
\usepackage{graphicx}
\usepackage{textcomp}
\usepackage{xcolor}
\usepackage{txfonts}
\usepackage{listings}
\usepackage{enumitem}
\usepackage{mathtools}
\usepackage{gensymb}
\usepackage{tfrupee}
\usepackage[breaklinks=true]{hyperref}
\usepackage{tkz-euclide} % loads  TikZ and tkz-base
\usepackage{listings}
%\usepackage{gvv}
%
%\usepackage{setspace}
%\usepackage{gensymb}
%\doublespacing
%\singlespacing

%\usepackage{graphicx}
%\usepackage{amssymb}
%\usepackage{relsize}
%\usepackage[cmex10]{amsmath}
%\usepackage{amsthm}
%\interdisplaylinepenalty=2500
%\savesymbol{iint}
%\usepackage{txfonts}
%\restoresymbol{TXF}{iint}
%\usepackage{wasysym}
%\usepackage{amsthm}
%\usepackage{iithtlc}
%\usepackage{mathrsfs}
%\usepackage{txfonts}
%\usepackage{stfloats}
%\usepackage{bm}
%\usepackage{cite}
%\usepackage{cases}
%\usepackage{subfig}
%\usepackage{xtab}
%\usepackage{longtable}
%\usepackage{multirow}
%\usepackage{algorithm}
%\usepackage{algpseudocode}
%\usepackage{enumitem}
%\usepackage{mathtools}
%\usepackage{tikz}
%\usepackage{circuitikz}
%\usepackage{verbatim}
%\usepackage{tfrupee}
%\usepackage{stmaryrd}
%\usetkzobj{all}
%    \usepackage{color}                                            %%
%    \usepackage{array}                                            %%
%    \usepackage{longtable}                                        %%
%    \usepackage{calc}                                             %%
%    \usepackage{multirow}                                         %%
%    \usepackage{hhline}                                           %%
%    \usepackage{ifthen}                                           %%
  %optionally (for landscape tables embedded in another document): %%
%    \usepackage{lscape}     
%\usepackage{multicol}
%\usepackage{chngcntr}
%\usepackage{enumerate}

%\usepackage{wasysym}
%\documentclass[conference]{IEEEtran}
%\IEEEoverridecommandlockouts
% The preceding line is only needed to identify funding in the first footnote. If that is unneeded, please comment it out.

\newtheorem{theorem}{Theorem}[section]
\newtheorem{problem}{Problem}
\newtheorem{proposition}{Proposition}[section]
\newtheorem{lemma}{Lemma}[section]
\newtheorem{corollary}[theorem]{Corollary}
\newtheorem{example}{Example}[section]
\newtheorem{definition}[problem]{Definition}
%\newtheorem{thm}{Theorem}[section] 
%\newtheorem{defn}[thm]{Definition}
%\newtheorem{algorithm}{Algorithm}[section]
%\newtheorem{cor}{Corollary}
\newcommand{\BEQA}{\begin{eqnarray}}
\newcommand{\EEQA}{\end{eqnarray}}
\newcommand{\define}{\stackrel{\triangle}{=}}
\theoremstyle{remark}
\newtheorem{rem}{Remark}

%\bibliographystyle{ieeetr}
\begin{document}
\title{LATEX ASSIGNMENT}
\author{ANAND}
\date{21-08-2023}
\maketitle
\section*{EXERCISE 10.3.4}
\begin{enumerate}
\item Solve the following pair of linear equations by the elimination method and the substitution method:
\begin{enumerate}[label=(\roman*)]
\item $x+y=5$ and $2x-3y=4$
\item $3x+4y=10$ and $ 2x-2y=2$
\item $3x-5y-4=0$ and $9x=2y+7$
\item $\frac{x}{2}+\frac{2y}{3}=-1$ and $x-\frac{y}{3}=0$
\end{enumerate}
\item Form the pair of linear equations in the following problem, and find their solutions(if they exist) by the elimination method:
\begin{enumerate}[label=(\roman*)]
\item If we add $1$ to the numerator and subtract $1$ from the denominator, a fraction reduces to $1$. It becomes $\frac{1}{2}$ if we only add $1$ to the denominator. What is the fraction?
\item Five years ago, Nuri was thrice as old as sonu. Ten years later, Nuri will be twice as old as sonu. How old are Nuri and sonu?
\item The Sum of the digits of a two-digit number is $9$. Also, nine times this number is twice the number obtained by reversing the order of the digits. Find the number.
\item Meena Went to a bank to withdraw \rupee~2000. She asked the cashier to give her \rupee~50 and \rupee~100 notes only. Meena got $25$ notes in all. Find how many notes \rupee~50 and \rupee~100 she received.
\item A lending library has a fixed charge for the first three days and an additional charge for each day thereafter. Sarita paid \rupee~27 for seven days, While susy paid \rupee~21 for the book she paid for five days. Find the fixed charge and the charge for each extra day.  
\end{enumerate}
\end{enumerate}
\end{document}   
