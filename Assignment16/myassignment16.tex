\let\negmedspace\undefined
\let\negthickspace\undefined
\documentclass{article}
\usepackage{cite}
\usepackage{amsmath,amssymb,amsfonts,amsthm}
\usepackage{algorithmic}
\usepackage{graphicx}
\usepackage{textcomp}
\usepackage{xcolor}
\usepackage{txfonts}
\usepackage{listings}
\usepackage{enumitem}
\usepackage{mathtools}
\usepackage{gensymb}
\usepackage[breaklinks=true]{hyperref}
\usepackage{tkz-euclide} % loads  TikZ and tkz-base
\usepackage{listings}
\usepackage{gvv}
%
%\usepackage{setspace}
%\usepackage{gensymb}
%\doublespacing
%\singlespacing

%\usepackage{graphicx}
%\usepackage{amssymb}
%\usepackage{relsize}
%\usepackage[cmex10]{amsmath}
%\usepackage{amsthm}
%\interdisplaylinepenalty=2500
%\savesymbol{iint}
%\usepackage{txfonts}
%\restoresymbol{TXF}{iint}
%\usepackage{wasysym}
%\usepackage{amsthm}
%\usepackage{iithtlc}
%\usepackage{mathrsfs}
%\usepackage{txfonts}
%\usepackage{stfloats}
%\usepackage{bm}
%\usepackage{cite}
%\usepackage{cases}
%\usepackage{subfig}
%\usepackage{xtab}
%\usepackage{longtable}
%\usepackage{multirow}
%\usepackage{algorithm}
%\usepackage{algpseudocode}
%\usepackage{enumitem}
%\usepackage{mathtools}
%\usepackage{tikz}
%\usepackage{circuitikz}
%\usepackage{verbatim}
%\usepackage{tfrupee}
%\usepackage{stmaryrd}
%\usetkzobj{all}
%    \usepackage{color}                                            %%
%    \usepackage{array}                                            %%
%    \usepackage{longtable}                                        %%
%    \usepackage{calc}                                             %%
%    \usepackage{multirow}                                         %%
%    \usepackage{hhline}                                           %%
%    \usepackage{ifthen}                                           %%
  %optionally (for landscape tables embedded in another document): %%
%    \usepackage{lscape}     
%\usepackage{multicol}
%\usepackage{chngcntr}
%\usepackage{enumerate}

%\usepackage{wasysym}
%\documentclass[conference]{IEEEtran}
%\IEEEoverridecommandlockouts
% The preceding line is only needed to identify funding in the first footnote. If that is unneeded, please comment it out.

\newtheorem{theorem}{Theorem}[section]
\newtheorem{problem}{Problem}
\newtheorem{proposition}{Proposition}[section]
\newtheorem{lemma}{Lemma}[section]
\newtheorem{corollary}[theorem]{Corollary}
\newtheorem{example}{Example}[section]
\newtheorem{definition}[problem]{Definition}
%\newtheorem{thm}{Theorem}[section] 
%\newtheorem{defn}[thm]{Definition}
%\newtheorem{algorithm}{Algorithm}[section]
%\newtheorem{cor}{Corollary}
\newcommand{\BEQA}{\begin{eqnarray}}
\newcommand{\EEQA}{\end{eqnarray}}
%\newcommand{\define}{\stackrel{\triangle}{=}}
\theoremstyle{remark}
\newtheorem{rem}{Remark}

%\bibliographystyle{ieeetr}

\begin{document}
\title{LATEX ASSIGNMENT}
\author{ANAND}
\date{2-09-2023}
\maketitle                                                                       >
\section*{EXERCISE 12.4.2}
Using the property of determnants and without expanding in \ref{prob:1} to \ref{prob:7}, prove that
\begin{enumerate}
\item $\mydet{ x&a&x+a\\ y&b&y+b\\ z&c&z+c}=0$ \label{prob:1}
\item $\mydet{ a-b&b-c&c-a\\ b-c&c-a&a-b\\ c-a&a-b&b-c}=0$
\item $\mydet{ 2&7&65\\ 3&8&75\\ 5&9&86}=0$
\item $\mydet{ 1&bc&a(b+c)\\ 1&ca&b(c+a)\\ 1&ab&c(a+b)}=0$
\item $\mydet{ b+c&q+r&y+z\\ c+a&r+p&z+x\\ a+b&p+q&x+y}=2$ $\mydet{a&p&x\\ b&q&y\\ c&r&z}$
\item $\mydet{ 0&a&-b\\ -a&0&-c\\ b&c&0}=0$
\item $\mydet{ -a^2&ab&ac\\ ba&-b^2&bc\\ ca&cb&-c^2} = -4a^2b^2c^2$ \label{prob:7}
\end{enumerate}
By using properties of determinants, in \ref{prob:8} to \ref{prob:14}, show that:
\begin{enumerate}[resume]
\item \label{prob:8}
\begin{enumerate}[label=(\roman*)]
\item $\mydet{1&a&a^2\\ 1&b&b^2\\ 1&c&c^2} = (a-b) (b-c) (c-a)$
\item $\mydet{1&1&1\\ a&b&c\\ a^3&b^3&c^3} = (a-b) (b-c) (c-a) (a+b+c)$
\end{enumerate}
\item $\mydet{x&x^2&yz\\  y&y^2&zx\\ z&z^2&xy} = (x-y) (y-z) (z-x) (xy+yz+zx)$
\item
\begin{enumerate}[label=(\roman*)]
\item $\mydet{x+4&2x&2x\\ 2x&x+4&2x\\ 2x&2x&x+4} = (5x+4)(4-x)^2$
\item $\mydet{y+k&y&y\\ y&y+k&y\\ y&y&y+k}=k^2(3y+k)$
\end{enumerate}
\item
\begin{enumerate}[label=(\roman*)]
\item $\mydet{a-b-c&2a&2a\\ 2b&b-c-a&2b\\ 2c&2c&c-a-b} = (a+b+c)^3$  
\item $\mydet{x+y+2z&x&y\\ z&y+z+2x&y\\ z&x&z+x+2y}= 2(x+y+z)^3$
\end{enumerate}
\item $\mydet{1&x&x^2\\ x^2&1&x\\ x&x^2&1} = (1-x^3)^2$
\item $\mydet{1+a^2-b^2&2ab&-2b\\ 2ab&1-a^2+b^2&2a\\ 2b&-2a&1-a^2-b^2} = (1+a^2+b^2)^3$
\item $\mydet{a^2+1&ab&ac\\ ab&b^2+1&bc\\ ca&cb&c^2+1} = 1+a^2+b^2+c^2$\label{prob:14}
\item Let $A$ be a square matrix of order $3 \times 3$, then $\abs{kA}$ is equal to 
\begin{enumerate}[label=(\roman*)]
\item $k \abs{A}$
\item $k^2 \abs{A}$
\item $k^3 \abs{A}$
\item $3k \abs{A}$
\end{enumerate}
\item Which of the following is correct 
\begin{enumerate}
\item Determinant is a square matrix
\item Determinant is a number associated to a matrix
\item Determinant is a number associated to a square matrix
\item None of these
\end{enumerate}
\end{enumerate}
\end{document}
