\let\negmedspace\undefined
\let\negthickspace\undefined
\documentclass{article}
\usepackage{cite}
\usepackage{amsmath,amssymb,amsfonts,amsthm}
\usepackage{algorithmic}
\usepackage{graphicx}
\usepackage{textcomp}
\usepackage{xcolor}
\usepackage{txfonts}
\usepackage{listings}
\usepackage{enumitem}
\usepackage{tfrupee}
\usepackage{mathtools}
\usepackage{gensymb}
\usepackage{tfrupee}
\usepackage[breaklinks=true]{hyperref}
\usepackage{tkz-euclide} % loads  TikZ and tkz-base
\usepackage{listings}
\usepackage{gvv}
%
%\usepackage{setspace}
%\usepackage{gensymb}
%\doublespacing
%\singlespacing

%\usepackage{graphicx}                                                                              >
%\usepackage{relsize}
%\usepackage[cmex10]{amsmath}                                                                        %\usepackage{amsthm}                                                                                 %\interdisplaylinepenalty=2500
%\savesymbol{iint}                                                                                   %\usepackage{txfonts}
%\restoresymbol{TXF}{iint}                                                                           %\usepackage{wasysym}
%\usepackage{amsthm}                                                                                 %\usepackage{iithtlc}
%\usepackage{mathrsfs}                                                                               %\usepackage{txfonts}
%\usepackage{stfloats}
%\usepackage{bm}
%\usepackage{cite}
%\usepackage{cases}
%\usepackage{subfig}
%\usepackage{xtab}
%\usepackage{longtable}
%\usepackage{multirow}
%\usepackage{algorithm}
%\usepackage{algpseudocode}
%\usepackage{enumitem}                                                                              >
%\usepackage{circuitikz}
%\usepackage{verbatim}
%\usepackage{tfrupee}
%\usepackage{stmaryrd}
%\usetkzobj{all}
%    \usepackage{color}                                            %%
%    \usepackage{array}                                            %%
%    \usepackage{longtable}                                        %%
%    \usepackage{calc}                                             %%
%    \usepackage{multirow}                                         %%
%    \usepackage{hhline}                                           %%
%    \usepackage{ifthen}                                           %%
  %optionally (for landscape tables embedded in another document): %%
%    \usepackage{lscape}
%\usepackage{multicol}
%\usepackage{chngcntr}
%\usepackage{enumerate}

%\usepackage{wasysym}                                                                               >
%\IEEEoverridecommandlockouts
% The preceding line is only needed to identify funding in the first footnote. If that is unneeded, >

\newtheorem{theorem}{Theorem}[section]
\newtheorem{problem}{Problem}
\newtheorem{proposition}{Proposition}[section]
\newtheorem{lemma}{Lemma}[section]
%\newtheorem{corollary}[theorem]{Corollary}                                                          >
\newtheorem{definition}[problem]{Definition}
%\newtheorem{thm}{Theorem}[section]
%\newtheorem{defn}[thm]{Definition}
%\newtheorem{algorithm}{Algorithm}[section]
%\newtheorem{cor}{Corollary}
\newcommand{\BEQA}{\begin{eqnarray}}
\newcommand{\EEQA}{\end{eqnarray}}
%\newcommand{\define}{\stackrel{\triangle}{=}}
\theoremstyle{remark}
\newtheorem{rem}{Remark}

%\bibliographystyle{ieeetr}

\begin{document}
\title{LATEX ASSIGNMENT}
\author{ANAND}
\date{29-08-2023}
\maketitle                                                                       >
\section*{EXERCISE 12.3.3}
\begin{enumerate}
\item Find the transpose of eaach of the following matrices:
\begin{enumerate}[label=(\roman*)]
\item $\myvec{ 5 \\ \frac{1}{2}\\ -1 }$
\item $\myvec {1 & -1\\ 2 & -3}$
\item $\myvec{-1 & 5 & 6\\ \sqrt{3} & 5 & 6\\ 2 & 3 & 1}$
\end{enumerate}
\item If $A=\myvec {-1 & 2 & 3\\ 5 & 7 & 9\\ -2 & 3 & 1}$ and $B= \myvec{-4 & 1 & -5\\ 1 & 2 & 0\\ 1 & 3 & 1}$
, then verify that
\begin{enumerate}
\item $(A+B)=A'+B'$
\item $(A-B)'=A'-B'$
\end{enumerate}
\item If $A=\myvec{3 & 4\\ -1 & 2\\ 0 & 1}$ and $B=\myvec{-1 & 0\\ 1 & 2}$, then find $(A+2B)'$
\item If $A=\myvec{-2 & 3\\ 1 & 2}$ and $B=\myvec{-1 & 0\\ 1 & 2} $, then find the $(A+2B)'$
\item For the matrices $A$ and $B$, Verify that $(AB)'= B'A'$, where 
\begin{enumerate}[label=(\roman*)]
\item $A=\myvec {1 \\ -4\\  3} , B=\myvec {-1.2 & 1}$
\item $A=\myvec {0  \\ 1 \\ 2} , B=\myvec {1 & 5 & 7}$
\end{enumerate}
\item If 
\begin{enumerate}[label=(\roman*)]
\item $A=\myvec {\cos \alpha &\sin \alpha \\ -\sin \alpha &\cos \alpha }$,
then verify that $A'A=1$
\item $A=\myvec{\sin \alpha &\cos \alpha \\ -\cos  \alpha & \sin \alpha}$
, then verify that $A'A=1$
\end{enumerate}
\item
\begin{enumerate}[label=(\roman*)]
\item Show that the matrix $A=\myvec{ 1 & -1 & 5\\ -1 & 2 & 1\\5 & 1 & 3} $ is a symmetrical matrix.
\item Show that the matrix $A=\myvec{ 0 &  1 & -1\\ -1 & 0 & 1\\ 1 & -1 & 0} $ is a skew symmetric matrix.
\end{enumerate} 
\item For the matrix $A=\myvec {1 & 5\\ 5 & 7}$, verify that
\begin{enumerate}[label=(\roman*)]
\item $(A+A)$ is a symmetric matrix.
\item $(A-A)$ ia a skew symmetric matrix.
\end{enumerate}
\item Find $\frac{1}{2} (A+A')$ and $\frac{1}{2} (A-A')$,
when $A=\myvec{ 0 & a & b\\ -a & 0 & c\\ -b & -c & 0}$
\item Express the following matrices as the sum of a symmetric and a skew symmetric matrix:
\begin{enumerate}[label=(\roman*)]
\item $\myvec{3 & 5\\ 1 & -1}$
\item $\myvec{6 & -2 & 2\\ -2 & 3 & -1\\ 2 & -1 & 3}$
\item $\myvec{3 & 3 & -1 \\ -2 & -2 & 1\\ -4 & -5 & 2}$
\item $\myvec{1 & 5\\ -1 & 2}$
\end{enumerate}
\item If $A$, $B$ are symmetric matrices of same order, then $AB-BA$ is a
\begin{enumerate}
\item Skew symmetric matrix
\item Symmetric matrix
\item Zero matrix
\item Identity matrix
\end{enumerate}
\item If $A=\myvec{\cos \alpha& -\sin \alpha \\ \sin \alpha &\cos \alpha}$ and $A+A'=1$ then the value of $\alpha$ is
\begin{enumerate}
\item $\frac{\pi}{6}$
\item $\frac{\pi}{3}$
\item $\pi$
\item $\frac{3\pi}{2}$
\end{enumerate}  
\end{enumerate}
\end{document}
