\let\negmedspace\undefined
\let\negthickspace\undefined
\documentclass{article}                                                                              \usepackage{cite}
\usepackage{amsmath,amssymb,amsfonts,amsthm}
\usepackage{algorithmic}                                                                             \usepackage{graphicx}
\usepackage{textcomp}
\usepackage{xcolor}
\usepackage{txfonts}
\usepackage{listings}
\usepackage{enumitem}
\usepackage{tfrupee}
\usepackage{mathtools}
\usepackage{gensymb}
\usepackage{tfrupee}
\usepackage[breaklinks=true]{hyperref}
\usepackage{tkz-euclide} % loads  TikZ and tkz-base
\usepackage{listings}
\usepackage{gvv}
%
%\usepackage{setspace}
%\usepackage{gensymb}
%\doublespacing
%\singlespacing

%\usepackage{graphicx}                                                                              >
%\usepackage{relsize}
%\usepackage[cmex10]{amsmath}                                                                       >
%\savesymbol{iint}                                                                                  >
%\restoresymbol{TXF}{iint}                                                                          >
%\usepackage{amsthm}                                                                                >
%\usepackage{mathrsfs}                                                                              >
%\usepackage{stfloats}
%\usepackage{bm}
%\usepackage{cite}
%\usepackage{cases}
%\usepackage{subfig}
%\usepackage{xtab}                                                                                   %\usepackage{longtable}
%\usepackage{multirow}                                                                               %\usepackage{algorithm}
%\usepackage{algpseudocode}                                                                          %\usepackage{enumitem}                                                                              >
%\usepackage{circuitikz}
%\usepackage{verbatim}
%\usepackage{tfrupee}
%\usepackage{stmaryrd}
%\usetkzobj{all}                                                                                     %    \usepackage{color}                                            %%
%    \usepackage{array}                                            %%
%    \usepackage{longtable}                                        %%
%    \usepackage{calc}                                             %%
%    \usepackage{multirow}                                         %%                                %    \usepackage{hhline}                                           %%
%    \usepackage{ifthen}                                           %%                                  %optionally (for landscape tables embedded in another document): %%
%    \usepackage{lscape}                                                                             %\usepackage{multicol}
%\usepackage{chngcntr}
%\usepackage{enumerate}

%\usepackage{wasysym}                                                                               >
%\IEEEoverridecommandlockouts
% The preceding line is only needed to identify funding in the first footnote. If that is unneeded, >

\newtheorem{theorem}{Theorem}[section]
\newtheorem{problem}{Problem}
\newtheorem{proposition}{Proposition}[section]
\newtheorem{lemma}{Lemma}[section]
%\newtheorem{corollary}[theorem]{Corollary}                                                         >
\newtheorem{definition}[problem]{Definition}
%\newtheorem{thm}{Theorem}[section]
%\newtheorem{defn}[thm]{Definition}
%\newtheorem{algorithm}{Algorithm}[section]
%\newtheorem{cor}{Corollary}
\newcommand{\BEQA}{\begin{eqnarray}}
\newcommand{\EEQA}{\end{eqnarray}}
%\newcommand{\define}{\stackrel{\triangle}{=}}
\theoremstyle{remark}
\newtheorem{rem}{Remark}

%\bibliographystyle{ieeetr}

\begin{document}
\title{LATEX ASSIGNMENT}
\author{ANAND}
\date{7-09-2023}
\maketitle
\section*{EXERCISE 12.11.2}
\begin{enumerate}
\item  Show that the three lines with direction cosines
$\frac{12}{13}, \frac{-3}{13}, \frac{-4}{13}$; $\frac{4}{13}, \frac{12}{13}, \frac{3}{13}$; $\frac{3}{13}, \frac{-4}{13}, \frac{12}{13}$; are mutually perpendicular.
\item  Show that the line through the points $(1,-1,2),(3,4,-2 )$ is perpendicular to the line through the points$(0,3,2)$ and$(3,5,6)$.
\item Show that the line through the points $(4,7,8),(2,3,4)$ is parallel to the line through the points $(-1,-2,1),(1,2,5)$.
\item  Find the equation of the line which passes through the point $(1,2,3)$ and is parallel to the vector $3\hat{i}+2\hat{j}-2\hat{k}$
\item  Find the equation of the line in vector and in cartesian form that passes through the point with position vector $2\hat{i}-\hat{j}+4\hat{k}$ and is in direction $\hat{i}+2\hat{j}-\hat{k}$.
\item Find the cartesian equation of the line which passes through the point $(-2,4,-5)$ and parallel to the line given by$ \frac{x+3}{3}=\frac{y-4}{5}=\frac{z+8}{6}$.
\item The cartesian equation of a line is $ \frac{x-5}{3}=\frac{y+4}{7}=\frac{z-6}{2}$. Write its vector form.
\item Find the vector and the cartesian equations of the lines that passes through the origin and $(5,-2,3)$
\item Find the vector and the cartesian equations of the line that passes through the points $(3,-2,-5),(3,-2,6)$.
\item  Find the angle between the following pairs of lines:
\begin{enumerate}[label=(\roman*)]	
\item  
\begin{align}
\overrightarrow{r}=2\hat{i}-5\hat{j}+\hat{k}+\lambda(3\hat{i}+2\hat{j}+6\hat{k}) \text{ and }\\ \overrightarrow{r}=7\hat{i}-6\hat{k}+\mu(\hat{i}+2\hat{j}+2\hat{k}) 
\end{align} 
\item 
\begin{align}
\overrightarrow{r}=3\hat{i}+\hat{j}-2\hat{k}+\lambda(\hat{i}-\hat{j}-2\hat{k}) \text{ and }\\ \overrightarrow{r}=2\hat{i}-\hat{j}-56\hat{k}+\mu(3\hat{i}-5\hat{j}-4\hat{k})
\end{align}
\end{enumerate}
\item Find the angle between the following pairs of lines:
\begin{enumerate}[label=(\roman*)]
\item 
\begin{align} \frac{x-2}{2}=\frac{y-1}{5}=\frac{z+3}{-3}\text{ and } \frac{x+2}{-1}=\frac{y-4}{8}=\frac{z-5}{4}.
\end{align}
\item
\begin{align} \frac{x}{2}=\frac{y}{2}=\frac{z}{1}\text{ and } \frac{x-5}{4}=\frac{y-2}{1}=\frac{z-3}{8}.
\end{align}
\end{enumerate}
\item Find the values of $p$ so that the lines $ \frac{1-x}{3}=\frac{7y-14}{2p}=\frac{z-3}{2}$ and $ \frac{7-7x}{3p}=\frac{y-5}{1}=\frac{6-z}{5}$ are at right angles.
\item Show that the lines $ \frac{x-5}{7}=\frac{y+2}{-5}=\frac{z}{1}$ and $ \frac{x}{1}=\frac{y}{2}=\frac{z}{3}$ are perpendicular to each other.
\item Find the shortest distance between the lines
\begin{align}
\overrightarrow{r}=(\hat{i}+2\hat{j}+\hat{k})+\lambda(\hat{i}-\hat{j}+\hat{k})\text{ and }\\ \overrightarrow{r}=2\hat{i}-\hat{j}-\hat{k}+\mu(2\hat{i}+\hat{j}+2\hat{k})
\end{align}
\item Find the shortest distance between the lines
\begin{align}
 \frac{x+1}{7}=\frac{y+1}{-6}=\frac{z+1}{1}\text{ and } \frac{x-3}{1}=\frac{y-5}{-2}=\frac{z-7}{1}
\end{align} 
\item Find the shortest distance between the lines whose vector equations are
\begin{align} 
\overrightarrow{r}=(\hat{i}+2\hat{j}+3\hat{k})+\lambda(\hat{i}-3\hat{j}+2\hat{k})\text{ and }\\ \overrightarrow{r}=4\hat{i}+5\hat{j}+6\hat{k}+\mu(2\hat{i}+3\hat{j}+\hat{k})
\end{align}
\item Find the shortest distance between the lines whose vector equations are 
\begin{align} 
\overrightarrow{r}=(1-t)\hat{i}+(t-2)\hat{j}+(3-2t)\hat{k} \text{ and }\\ \overrightarrow{r}=(s+1)\hat{i}+(2s-1)\hat{j}-(2s+1)\hat{k}
\end{align}
\end{enumerate}
\end{document}
