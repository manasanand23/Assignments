\let\negmedspace\undefined
\let\negthickspace\undefined
\documentclass{article}
\usepackage{cite}
\usepackage{amsmath,amssymb,amsfonts,amsthm}
\usepackage{algorithmic}
\usepackage{graphicx}
\usepackage{textcomp}
\usepackage{xcolor}
\usepackage{txfonts}
\usepackage{listings}
\usepackage{enumitem}
\usepackage{tfrupee}
\usepackage{mathtools}
\usepackage{gensymb}
\usepackage{tfrupee}
\usepackage[breaklinks=true]{hyperref}
\usepackage{tkz-euclide} % loads  TikZ and tkz-base
\usepackage{listings}
\usepackage{gvv}                                                                                                                             %
%\usepackage{setspace}
%\usepackage{gensymb}
%\doublespacing
%\singlespacing

%\usepackage{graphicx}                                                                               %\usepackage{amssymb}
%\usepackage{relsize}
%\usepackage[cmex10]{amsmath}
%\usepackage{amsthm}
%\interdisplaylinepenalty=2500
%\savesymbol{iint}
%\usepackage{txfonts}
%\restoresymbol{TXF}{iint}
%\usepackage{wasysym}
%\usepackage{amsthm}
%\usepackage{iithtlc}
%\usepackage{mathrsfs}
%\usepackage{txfonts}
%\usepackage{stfloats}
%\usepackage{bm}
%\usepackage{cite}
%\usepackage{cases}
%\usepackage{subfig}
%\usepackage{xtab}
%\usepackage{longtable}
%\usepackage{multirow}
%\usepackage{algorithm}
%\usepackage{algpseudocode}
%\usepackage{enumitem}                                                                               %\usepackage{mathtools}                 >
%\usepackage{circuitikz}
%\usepackage{verbatim}
%\usepackage{tfrupee}
%\usepackage{stmaryrd}
%\usetkzobj{all}
%    \usepackage{color}                                            %%
%    \usepackage{array}                                            %%
%    \usepackage{longtable}                                        %%
%    \usepackage{calc}                                             %%
%    \usepackage{multirow}                                         %%
%    \usepackage{hhline}                                           %%
%    \usepackage{ifthen}                                           %%
  %optionally (for landscape tables embedded in another document): %%
%    \usepackage{lscape}
%\usepackage{multicol}
%\usepackage{chngcntr}
%\usepackage{enumerate}

%\usepackage{wasysym}                                                                                %\documentclass[conference]{IEEEtran}    %\IEEEoverridecommandlockouts
% The preceding line is only needed to identify funding in the first footnote. If that is unneeded, >

\newtheorem{theorem}{Theorem}[section]
\newtheorem{problem}{Problem}
\newtheorem{proposition}{Proposition}[section]
\newtheorem{lemma}{Lemma}[section]
\newtheorem{corollary}[theorem]{Corollary}                                                           \newtheorem{example}{Example}[section]
\newtheorem{definition}[problem]{Definition}
%\newtheorem{thm}{Theorem}[section]
%\newtheorem{defn}[thm]{Definition}
%\newtheorem{algorithm}{Algorithm}[section]
%\newtheorem{cor}{Corollary}
\newcommand{\BEQA}{\begin{eqnarray}}
\newcommand{\EEQA}{\end{eqnarray}}
%\newcommand{\define}{\stackrel{\triangle}{=}}
\theoremstyle{remark}
\newtheorem{rem}{Remark}

%\bibliographystyle{ieeetr}

\begin{document}
\title{LATEX ASSIGNMENT}
\author{ANAND}
\date{2-09-2023}
\maketitle                                                                       >
\section*{EXERCISE 12.3.5}
\begin{enumerate}
\item Let $A=\myvec{0 & 1\\ 0 & 0}$, show that $(aI+bA)^n= a^nI+na^{n-1} bA$, where $I$ is the identity matrix of order 2 and $n \in N$.
\item If $A=\myvec{1 & 1 & 1\\ 1 & 1 & 1\\ 1 & 1 & 1}$, Prove that $A^n=\myvec{3^{n-1} & 3^{n-1} & 3^{n-1}\\ 3^{n-1} & 3^{n-1} & 3^{n-1}\\ 3^{n-1} & 3^{n-1} & 3^{n-1}}, n \in N$.
\item If $A=\myvec{3 & -4\\ 1 & -1}$, then prove that $A^n=\myvec{1+2n & -4n\\ n & 1-2n}$,Where $n$ is any positive integer.
\item If $A$ and $B$ are symmetric matrices prove that $AB-BA$ is a skew symmetric matrix.
\item Show that the matrix  $B^{\prime}AB$ is a symmetric or skew symmetric according as $A$ is symmetric or skew symmetric.
\item Find the value of $x,y,z$ if the matrix $A=\myvec{0 & 2y & z\\ x & y & -z\\x & -y & z}$ satisfy the equation $A^{\prime}A=I$.
\item For what values  of $x : \myvec{1&2&1} \myvec{1&2&0\\ 2&0&1\\ 1&0&2}$ $\myvec{0\\2\\x}=0$
\item If $A=\myvec{3 & 1\\ -1 & 2}$, show that $A^2-5A+7I=0$.
\item Find $x$, if $\myvec{x & -5 & -1} \myvec{1 & 0 & 2\\ 0 & 2 & 1\\ 2 & 0 & 3} \myvec{x\\ 4\\ 1}=0$. 
\item A manufacturer produces three products $x, y, z$ which he sells in two markets. Annual Sales are indicated below:
\begin{table}
\centering
\begin{tabular}{|c|c c c|}
\hline
market & products\\
\hline
I &10,000 &2,000 &18,000\\
\hline
II &6,000 &20,000 &8,000\\
\hline
\end{tabular}
\caption{}
\end{table}
\begin{enumerate}
\item If unit sale Prices of $x$, $y$ and $z$ are \rupee~2.50, \rupee~1.50 and \rupee~1.00, respectively. Find the total revenue in each market with the help of matrix algebra.
\item If the unit costs of the above three commodities are \rupee~2.00, \rupee~1.00 and $50$ paise respectively. Find the gross profit.
\end{enumerate}
\item Find the matrix $X$ so that $X\myvec{1&2&3\\ 4&5&6}$= $\myvec{-7&-8&-9\\ 2&4&6}$
\item If $A$ and $B$ are square matrices of the same order such that $AB=BA$, then prove by induction that $(AB)^n=B^nA^n$. Further prove that $(AB)^n=A^nB^n$ for all $n \in N$. Choose the correct answer in the following questions:
\item If $A=\myvec{\alpha& \beta \\ \gamma& -\alpha}$ is such that $A^2= I$, then
\begin{enumerate} 
\item $1+ \alpha^2+ \beta \gamma=0$
\item $1-\alpha^2+ \beta \gamma=0$
\item $1-\alpha^2-\beta \gamma=0$
\item $1+\alpha^2- \beta \gamma=0$ 
\end{enumerate}
\item If the matrix $A$ is both symmetric and skew symmetric, then
\begin{enumerate}
\item $A$ is a diagonal matrix
\item $A$ is a Zero matrix
\item $A$ is a Square matrix
\item None of these
\end{enumerate}
\item If $A$ is square matrix such that $A^2=A$, then $(I+A)^3-7A$ is equal to
\begin{enumerate}
\item $A$
\item $I-A$
\item $I$
\item $3A$
\end{enumerate}
\end{enumerate}
\end{document}

