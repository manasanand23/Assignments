\let\negmedspace\undefined
\let\negthickspace\undefined
\documentclass{article}
\usepackage{cite}
\usepackage{amsmath,amssymb,amsfonts,amsthm}
\usepackage{algorithmic}
\usepackage{graphicx}
\usepackage{textcomp}
\usepackage{xcolor}
\usepackage{txfonts}
\usepackage{listings}
\usepackage{enumitem}
\usepackage{mathtools}
\usepackage{gensymb}
\usepackage{tfrupee}
\usepackage[breaklinks=true]{hyperref}
\usepackage{tkz-euclide} % loads  TikZ and tkz-base
\usepackage{listings}
%\usepackage{gvv}
%
%\usepackage{setspace}
%\usepackage{gensymb}
%\doublespacing
%\singlespacing

%\usepackage{graphicx}
%\usepackage{amssymb}
%\usepackage{relsize}
%\usepackage[cmex10]{amsmath}
%\usepackage{amsthm}
%\interdisplaylinepenalty=2500
%\savesymbol{iint}
%\usepackage{txfonts}
%\restoresymbol{TXF}{iint}
%\usepackage{wasysym}
%\usepackage{amsthm}
%\usepackage{iithtlc}
%\usepackage{mathrsfs}
%\usepackage{txfonts}
%\usepackage{stfloats}
%\usepackage{bm}
%\usepackage{cite}
%\usepackage{cases}
%\usepackage{subfig}
%\usepackage{xtab}
%\usepackage{longtable}
%\usepackage{multirow}
%\usepackage{algorithm}
%\usepackage{algpseudocode}
%\usepackage{enumitem}
%\usepackage{mathtools}
%\usepackage{tikz}
%\usepackage{circuitikz}
%\usepackage{verbatim}
%\usepackage{tfrupee}
%\usepackage{stmaryrd}
%\usetkzobj{all}
%    \usepackage{color}                                            %%
%    \usepackage{array}                                            %%
%    \usepackage{longtable}                                        %%
%    \usepackage{calc}                                             %%
%    \usepackage{multirow}                                         %%
%    \usepackage{hhline}                                           %%
%    \usepackage{ifthen}                                           %%
  %optionally (for landscape tables embedded in another document): %%
%    \usepackage{lscape}     
%\usepackage{multicol}
%\usepackage{chngcntr}
%\usepackage{enumerate}

%\usepackage{wasysym}
%\documentclass[conference]{IEEEtran}
%\IEEEoverridecommandlockouts
% The preceding line is only needed to identify funding in the first footnote. If that is unneeded, please comment it out.

\newtheorem{theorem}{Theorem}[section]
\newtheorem{problem}{Problem}
\newtheorem{proposition}{Proposition}[section]
\newtheorem{lemma}{Lemma}[section]
\newtheorem{corollary}[theorem]{Corollary}
\newtheorem{example}{Example}[section]
\newtheorem{definition}[problem]{Definition}
%\newtheorem{thm}{Theorem}[section] 
%\newtheorem{defn}[thm]{Definition}
%\newtheorem{algorithm}{Algorithm}[section]
%\newtheorem{cor}{Corollary}
\newcommand{\BEQA}{\begin{eqnarray}}
\newcommand{\EEQA}{\end{eqnarray}}
\newcommand{\define}{\stackrel{\triangle}{=}}
\theoremstyle{remark}
\newtheorem{rem}{Remark}

%\bibliographystyle{ieeetr}
\begin{document}
\title{LATEX ASSIGNMENT}
\author{ANAND}
\date{19-08-2023}
\maketitle
\section*{EXERCISE 10.3.6}
\begin{enumerate}
\item Solve the following pair of equations by reducing them to a pair of linear equations:
\begin{enumerate}[label=(\roman*)]
\item
\begin{align}
\frac{1}{2x}+\frac{1}{3y}=2 \\ \frac{1}{3x}+\frac{1}{2y}=\frac{13}{6}
\end{align}
\item
\begin{align}
\frac{2}{\sqrt{x}}+\frac{3}{\sqrt{y}}=2\\
\frac{4}{\sqrt{x}}-\frac{9}{\sqrt{y}}=-1
\end{align}
\item
\begin{align}
\frac{4}{x}+3y=14\\ \frac{3}{x}-4y=23
\end{align}
\item
\begin{align}
\frac{5}{x-1}+\frac{1}{y-2}=2\\ \frac{6}{x-1}-\frac{3}{y-2}=1
\end{align}
\item
\begin{align}
\frac{7x-2y}{xy}=5\\ \frac{8x+7y}{xy}=15
\end{align}
\item
\begin{align}
6x+3y=6xy\\ 2x+4y=5xy
\end{align}
\item
\begin{align}
\frac{10}{x+y}+\frac{2}{x-y}=4\\ \frac{15}{x+y}-\frac{5}{x-y}=-2
\end{align}
\item
\begin{align}
\frac{1}{3x+y}+\frac{1}{3x-y}=\frac{3}{4}\\ \frac{1}{2(3x+y)}-\frac{1}{2(3x-y)}=\frac{-1}{8}
\end{align}
\end{enumerate}
\item Formulate the following problems as a pair of equations,and hence find their solutions:
\begin{enumerate}[label=(\roman*)]
\item Ritu can row downstream $20km$ in $2$ hours,and upstream $4km$ in $2$ hours.Find her speed of rowing in still water and the speed of the current.
\item $2$ women and $5$ men can together finish an embroidery work in $4$ days, while $3$ women and $6$ men can finish it in $3$ days.Find the time taken by $1$ women along to finish the work,and also that taken by $1$ men alone.
\item Roohi travels $300 km$ to her home partly by train and partly by bus.She takes $4$ hours if she travels $60km$ by train and the remaining by bus.If she travels $100km$ by train and the remaining by bus, she takes $10$ minutes longer.Find the speed of the train and the bus seperately
\end{enumerate}
\end{enumerate}
\end{document} 
